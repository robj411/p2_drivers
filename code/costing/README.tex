% Options for packages loaded elsewhere
\PassOptionsToPackage{unicode}{hyperref}
\PassOptionsToPackage{hyphens}{url}
%
\documentclass[
]{article}
\usepackage{amsmath,amssymb}
\usepackage{iftex}
\ifPDFTeX
  \usepackage[T1]{fontenc}
  \usepackage[utf8]{inputenc}
  \usepackage{textcomp} % provide euro and other symbols
\else % if luatex or xetex
  \usepackage{unicode-math} % this also loads fontspec
  \defaultfontfeatures{Scale=MatchLowercase}
  \defaultfontfeatures[\rmfamily]{Ligatures=TeX,Scale=1}
\fi
\usepackage{lmodern}
\ifPDFTeX\else
  % xetex/luatex font selection
\fi
% Use upquote if available, for straight quotes in verbatim environments
\IfFileExists{upquote.sty}{\usepackage{upquote}}{}
\IfFileExists{microtype.sty}{% use microtype if available
  \usepackage[]{microtype}
  \UseMicrotypeSet[protrusion]{basicmath} % disable protrusion for tt fonts
}{}
\makeatletter
\@ifundefined{KOMAClassName}{% if non-KOMA class
  \IfFileExists{parskip.sty}{%
    \usepackage{parskip}
  }{% else
    \setlength{\parindent}{0pt}
    \setlength{\parskip}{6pt plus 2pt minus 1pt}}
}{% if KOMA class
  \KOMAoptions{parskip=half}}
\makeatother
\usepackage{xcolor}
\usepackage[margin=1in]{geometry}
\usepackage{longtable,booktabs,array}
\usepackage{calc} % for calculating minipage widths
% Correct order of tables after \paragraph or \subparagraph
\usepackage{etoolbox}
\makeatletter
\patchcmd\longtable{\par}{\if@noskipsec\mbox{}\fi\par}{}{}
\makeatother
% Allow footnotes in longtable head/foot
\IfFileExists{footnotehyper.sty}{\usepackage{footnotehyper}}{\usepackage{footnote}}
\makesavenoteenv{longtable}
\usepackage{graphicx}
\makeatletter
\def\maxwidth{\ifdim\Gin@nat@width>\linewidth\linewidth\else\Gin@nat@width\fi}
\def\maxheight{\ifdim\Gin@nat@height>\textheight\textheight\else\Gin@nat@height\fi}
\makeatother
% Scale images if necessary, so that they will not overflow the page
% margins by default, and it is still possible to overwrite the defaults
% using explicit options in \includegraphics[width, height, ...]{}
\setkeys{Gin}{width=\maxwidth,height=\maxheight,keepaspectratio}
% Set default figure placement to htbp
\makeatletter
\def\fps@figure{htbp}
\makeatother
\setlength{\emergencystretch}{3em} % prevent overfull lines
\providecommand{\tightlist}{%
  \setlength{\itemsep}{0pt}\setlength{\parskip}{0pt}}
\setcounter{secnumdepth}{5}
\usepackage{float}
\usepackage{booktabs}
\usepackage{longtable}
\usepackage{array}
\usepackage{multirow}
\usepackage{wrapfig}
\usepackage{colortbl}
\usepackage{pdflscape}
\usepackage{tabu}
\usepackage{threeparttable}
\usepackage{threeparttablex}
\usepackage[normalem]{ulem}
\usepackage{makecell}
\usepackage{xcolor}
\ifLuaTeX
  \usepackage{selnolig}  % disable illegal ligatures
\fi
\usepackage[]{natbib}
\bibliographystyle{plainnat}
\usepackage{bookmark}
\IfFileExists{xurl.sty}{\usepackage{xurl}}{} % add URL line breaks if available
\urlstyle{same}
\hypersetup{
  pdftitle={The Costing Model},
  hidelinks,
  pdfcreator={LaTeX via pandoc}}

\title{The Costing Model}
\author{}
\date{\vspace{-2.5em}}

\begin{document}
\maketitle

This document describes the costing model that is used in the CEPI application.

\section{Parameters}\label{parameters}

\begin{longtable}[]{@{}
  >{\centering\arraybackslash}p{(\columnwidth - 10\tabcolsep) * \real{0.1325}}
  >{\centering\arraybackslash}p{(\columnwidth - 10\tabcolsep) * \real{0.1988}}
  >{\centering\arraybackslash}p{(\columnwidth - 10\tabcolsep) * \real{0.1145}}
  >{\centering\arraybackslash}p{(\columnwidth - 10\tabcolsep) * \real{0.1928}}
  >{\centering\arraybackslash}p{(\columnwidth - 10\tabcolsep) * \real{0.1325}}
  >{\centering\arraybackslash}p{(\columnwidth - 10\tabcolsep) * \real{0.2289}}@{}}
\caption{Notation and parametric assumptions for inputs to the costing model. Parameters are used as follows: uniform distributions go from Parameter 1 to Parameter 2. Triangular distributions go from Parameter 1 to Parameter 3 with a peak at Parameter 2. Multinomial distributions have equally probable values listed individually. Exponential distributions have as a mean Parameter 1. Inverse Gaussian distributions have as a mean Parameter 1, and as a shape Parameter 2. Log normal distributions have as a mean Parameter 1, and as a standard deviation Parameter 2. Inverse Gamma distributions have shape Parameter 1 and scale Parameter 2. Beta Prime distributions have shape Parameters 1 and 2, and scale Parameter 3. Where given, distributions are truncated at bounds.}\tabularnewline
\toprule\noalign{}
\begin{minipage}[b]{\linewidth}\centering
Math notation
\end{minipage} & \begin{minipage}[b]{\linewidth}\centering
Description
\end{minipage} & \begin{minipage}[b]{\linewidth}\centering
Distribution
\end{minipage} & \begin{minipage}[b]{\linewidth}\centering
Parameters
\end{minipage} & \begin{minipage}[b]{\linewidth}\centering
Bounds
\end{minipage} & \begin{minipage}[b]{\linewidth}\centering
Source
\end{minipage} \\
\midrule\noalign{}
\endfirsthead
\toprule\noalign{}
\begin{minipage}[b]{\linewidth}\centering
Math notation
\end{minipage} & \begin{minipage}[b]{\linewidth}\centering
Description
\end{minipage} & \begin{minipage}[b]{\linewidth}\centering
Distribution
\end{minipage} & \begin{minipage}[b]{\linewidth}\centering
Parameters
\end{minipage} & \begin{minipage}[b]{\linewidth}\centering
Bounds
\end{minipage} & \begin{minipage}[b]{\linewidth}\centering
Source
\end{minipage} \\
\midrule\noalign{}
\endhead
\bottomrule\noalign{}
\endlastfoot
\(W_{0; 365}^{(S)}\) & SSV preclinical duration
(365); weeks & Constant & 14 & & \\
\(W_{0; 200}^{(S)}\) & SSV preclinical duration
(200DM); weeks & Constant & 5 & & \\
\(W_{0; 100}^{(S)}\) & SSV preclinical duration
(100DM); weeks & Constant & 5 & & \\
\(W_{1; 365}^{(S)}\) & SSV phase I duration (365);
weeks & Constant & 19 & & \\
\(W_{1; 200}^{(S)}\) & SSV phase I duration (200DM);
weeks & Constant & 7 & & \\
\(W_{1; 100}^{(S)}\) & SSV phase I duration (100DM);
weeks & Constant & 0 & & \\
\(W_{2; 365}^{(S)}\) & SSV phase II duration (365);
weeks & Constant & 19 & & \\
\(W_{2; 200}^{(S)}\) & SSV phase II duration (200DM);
weeks & Constant & 0 & & \\
\(W_{2; 100}^{(S)}\) & SSV phase II duration (100DM);
weeks & Constant & 0 & & \\
\(W_{3; 365}^{(S)}\) & SSV phase III duration (365);
weeks & Constant & 16 & & \\
\(W_{3; 200}^{(S)}\) & SSV phase III duration
(200DM); weeks & Constant & 15 & & \\
\(W_{3; 100}^{(S)}\) & SSV phase III duration
(100DM); weeks & Constant & 8 & & \\
\(V_{L; 0}\) & Cost of vaccine delivery at
start up (0--10\%) in LIC;
USD per dose & Triangular & 1, 1.5, 2 & & See Table \ref{tab:delcosts} \\
\(V_{L; 11}\) & Cost of vaccine delivery
during ramp up (11--30\%) in
LIC; USD per dose & Triangular & 0.75, 1, 1.5 & & See Table \ref{tab:delcosts} \\
\(V_{L; 31}\) & Cost of vaccine delivery
getting to scale (31--80\%)
in LIC; USD per dose & Triangular & 1, 2, 4 & & See Table \ref{tab:delcosts} \\
\(V_{LM; 0}\) & Cost of vaccine delivery at
start up (0--10\%) in LMIC;
USD per dose & Triangular & 3, 4.5, 6 & & See Table \ref{tab:delcosts} \\
\(V_{LM; 11}\) & Cost of vaccine delivery
during ramp up (11--30\%) in
LMIC; USD per dose & Triangular & 2.25, 3, 4.5 & & See Table \ref{tab:delcosts} \\
\(V_{LM; 31}\) & Cost of vaccine delivery
getting to scale (31--80\%)
in LMIC; USD per dose & Triangular & 1.5, 2, 2.5 & & See Table \ref{tab:delcosts} \\
\(V_{UM; 0}\) & Cost of vaccine delivery at
start up (0--10\%) in UMIC;
USD per dose & Triangular & 6, 9, 12 & & See Table \ref{tab:delcosts} \\
\(V_{UM; 11}\) & Cost of vaccine delivery
during ramp up (11--30\%) in
UMIC; USD per dose & Triangular & 4.5, 6, 9 & & See Table \ref{tab:delcosts} \\
\(V_{UM; 31}\) & Cost of vaccine delivery
getting to scale (31--80\%)
in UMIC; USD per dose & Triangular & 3, 4, 5 & & See Table \ref{tab:delcosts} \\
\(V_{H; 0}\) & Cost of vaccine delivery at
start up (0--10\%) in HIC;
USD per dose & Triangular & 30, 40, 75 & & See Table \ref{tab:delcosts} \\
\(V_{H; 11}\) & Cost of vaccine delivery
during ramp up (11--30\%) in
HIC; USD per dose & Triangular & 30, 40, 75 & & See Table \ref{tab:delcosts} \\
\(V_{H; 31}\) & Cost of vaccine delivery
getting to scale (31--80\%)
in HIC; USD per dose & Triangular & 30, 40, 75 & & See Table \ref{tab:delcosts} \\
\(M_G\) & Global annual manufacturing
volume; billion doses & Constant & 15 & & \citet{LinksbridgeSPC2025} \\
\(M_C\) & Current annual manufacturing
volume; billion doses & Constant & 6.6 & & \citet{LinksbridgeSPC2025} \\
\(F\) & Facility transition start;
weeks before vaccine approval & Constant & 7 & & \\
\(I_R\) & Weeks to initial
manufacturing, reserved
infrastructure & Constant & 12 & & \citet{VaccinesEurope2023} \\
\(I_E\) & Weeks to initial
manufacturing, existing and
unreserved infrastructure & Constant & 30 & & \citet{VaccinesEurope2023} \\
\(I_B\) & Weeks to initial
manufacturing, built and
unreserved infrastructure & Constant & 48 & & \\
\(C_R\) & Weeks to scale up to full
capacity, reserved
infrastructure & Constant & 10 & & \citet{VaccinesEurope2023} \\
\(C_E\) & Weeks to scale up to full
capacity, existing and
unreserved infrastructure & Constant & 16 & & \\
\(C_B\) & Weeks to scale up to full
capacity, built and unreserved
infrastructure & Constant & 16 & & \\
\(P_0\) & Probability of success;
preclinical & Multinomial & 0.40, 0.41, 0.41, 0.42, 0.48,
0.57 & & \citet{Gouglas2018} \\
\(P_1\) & Probability of success; Phase
I & Multinomial & 0.33, 0.40, 0.50, 0.68, 0.70,
0.72, 0.74, 0.77, 0.81, 0.90 & & \citet{Gouglas2018} \\
\(P_2\) & Probability of success; Phase
II & Multinomial & 0.22, 0.31, 0.33, 0.43, 0.46,
0.54, 0.58, 0.58, 0.74, 0.79 & & \citet{Gouglas2018} \\
\(P_3\) & Probability of success; Phase
III & Uniform & 0.4, 0.8 & & \citet{Wong2019} \\
\(T_0^{(e)}\) & Cost, preclinical, experienced
manufacturer; USD & Exponential & 24213683 & 1700000, 140000000 & \citet{Gouglas2018} \\
\(T_0^{(n)}\) & Cost, preclinical,
inexperienced manufacturer;
USD & Inverse Gaussian & 7882792, 13455907 & 1700000, 37000000 & \citet{Gouglas2018} \\
\(T_1^{(e)}\) & Cost, Phase I, experienced
manufacturer; USD & Inverse Gaussian & 15339198, 8076755 & 1900000, 70000000 & \citet{Gouglas2018} \\
\(T_1^{(n)}\) & Cost, Phase I, inexperienced
manufacturer; USD & Inverse Gamma & 2.2774, 9799081 & 1000000, 30000000 & \citet{Gouglas2018} \\
\(T_2^{(e)}\) & Cost, Phase II, experienced
manufacturer; USD & Log normal & 28297339, 24061641 & 3800000, 140000000 & \citet{Gouglas2018} \\
\(T_2^{(n)}\) & Cost, Phase II, inexperienced
manufacturer; USD & Inverse Gaussian & 17124622, 35918793 & 4400000, 54000000 & \citet{Gouglas2018} \\
\(T_3^{(e)}\) & Cost, Phase III, experienced
manufacturer; USD & Inverse Gamma & 1.3147, 51397313 & 15000000, 910000000 & \citet{Gouglas2018} \\
\(T_3^{(n)}\) & Cost, Phase III, inexperienced
manufacturer; USD & Beta prime & 4.8928, 1.6933, 11400026 & 2500000, 400000000 & \citet{Gouglas2018} \\
\(\omega\) & Share of manufacturers that
are inexperienced & Constant & 0.9 & & \\
\(L\) & Licensure; USD & Constant & 287750 & & \citet{Gouglas2018} \\
\(Y_0^{(B)}\) & BPSV preclinical duration;
years & Multinomial & 1, 2 & & \citet{CEPI2022} \\
\(Y_1^{(B)}\) & BPSV Phase I duration; years & Multinomial & 1, 2 & & \citet{CEPI2022} \\
\(Y_2^{(B)}\) & BPSV Phase II duration; years & Constant & 2 & & \citet{CEPI2022} \\
\(Y_3^{(B)}\) & BPSV Phase III duration; weeks & Constant & 18 & & \\
\(L^{(B)}\) & Licensure duration; years & Constant & 2 & & \citet{CEPI2022} \\
\(G\) & BPSV cost of goods supplied;
USD per dose & Constant & 4.68 & & \citet{Kazaz2021} \\
\(A_2\) & Advanced capacity reservation
fee; USD per dose per year & Constant & 0.53 & & \citet{Pfizer2023} \\
\(A_1\) & Stockpiling fee; USD per dose
per year & Constant & 2 & & \\
\(S_R\) & SSV procurement price,
reserved capacity; USD per
dose & Constant & 6.29 & & \citet{Kazaz2021} \\
\(S_U\) & SSV procurement price,
reactive capacity; USD per
dose & Constant & 18.94 & & \citet{LinksbridgeSPC2025} \\
\(E\) & Enabling activities; million
USD per year & Constant & 700 & & \citet{CEPI2021} \\
\(I\) & Inflation (2018--2025) & Constant & 0.28 & & \citet{U.S.BureauofLaborStatistics} \\
\(r\) & Discount rate & Uniform & 0.02, 0.06 & & \citet{Glennerster2023} \\
\(M_p\) & Profit margin & Constant & 0.2 & & \\
\(M_f\) & Fill/finish cost & Constant & 0.14 & & \\
\(N_{HIC}^{(15)}\) & Population aged 15 and older,
HIC & Constant & 1062903718 & & \citet{OWID2024} \\
\(N_{UMIC}^{(15)}\) & Population aged 15 and older,
UMIC & Constant & 2258682374 & & \citet{OWID2024} \\
\(N_{LMIC}^{(15)}\) & Population aged 15 and older,
LMIC & Constant & 2292686818 & & \citet{OWID2024} \\
\(N_{LIC}^{(15)}\) & Population aged 15 and older,
LIC & Constant & 431149981 & & \citet{OWID2024} \\
\(N_{HIC}^{(65)}\) & Population aged 65 and older,
HIC & Constant & 245880785 & & \citet{OWID2024} \\
\(N_{UMIC}^{(65)}\) & Population aged 65 and older,
UMIC & Constant & 340100977 & & \citet{OWID2024} \\
\(N_{LMIC}^{(65)}\) & Population aged 65 and older,
LMIC & Constant & 196323876 & & \citet{OWID2024} \\
\(N_{LIC}^{(65)}\) & Population aged 65 and older,
LIC & Constant & 23832449 & & \citet{OWID2024} \\
\(N^{\text{(SSV)}}\) & Number of SSV candidates & Constant & 18 & & \\
\(N^{\text{(BPSV)}}\) & Number of BPSV candidates & Constant & 8 & & \citet{CEPI2025} \\
\(A_3\) & Reserved capacity for HIC,
billions & Constant & 0.5 & & \\
\(\lambda\) & Final vaccine coverage,
proportion of population & Constant & 0.8 & & Model choice \\
\end{longtable}

\section{Preparedness cost equation}\label{preparedness-cost-equation}

{ (BPSV R\&D + BPSV Stockpile + SARS-X Reserved capacity + Enabling activities) / (1 + discount rate) \^{} (year -- 2025) }

\[D_y^{\text{(prep)}} = \frac{1}{(1+r)^y}\left(D_s^{\text{(BP-adRD)}} + D_{s,y}^{\text{(BP-inv)}} + D_s^{\text{(S-cap)}} + D_{s,y}^{\\text{(en)}}\right)\]

\begin{itemize}
\tightlist
\item
  \(D_s^{\text{(BP-adRD)}}\) is the R\&D cost of BPSV prior to an outbreak; see Equation \eqref{eq:bpsvrd}
\item
  \(D_{s,y}^{\text{(BP-inv)}}\) is the cost of maintaining an investigational reserve of 100,000 BPSV doses; see Equation \eqref{eq:bpsvinv}
\item
  \(D_s^{\text{(S-cap)}}\) is the cost of reserved capacity for SSV; see Equation \eqref{eq:ssvcap}
\item
  \(D_{s,y}^{\\text{(en)}}\) is the annual cost of enabling activities; see Equation \eqref{eq:enable}.
\end{itemize}

\subsection{BPSV advanced R\&D}\label{bpsv-advanced-rd}

\textbf{I have set the weight of inexperienced manufacturer to 0.9}

Probabilities of success for preclinical, Phase I, Phase II, and Phase III are \(P_0\), \(P_1\), \(P_2\) and \(P_3\). Then probabilities of occurrence are:

\[
\hat{P}_i = \left\{\begin{array}{lr}1 & i=0 \\ \prod_{j=0}^{i-1}P_j & i\in\{1,2,3\} \\ \prod_{j=0}^{3}P_j & i=L \end{array}\right.
\]

and the cost of each phase is \(T_i\), a weighted average of experienced and inexperienced manufacturers (assuming \(\omega=0.9\)):

\[T_{i} = \omega T_i^{(n)} + (1-\omega)T_i^{(e)}.\] Then the total weighted cost for phases 0 through 2 for \(N^{\text{(BPSV)}} = 8\) candidates is

\begin{equation}
D_s^{\text{(BP-adRD)}} = \left\{\begin{array}{lr}
 N^{\text{(BPSV)}}\sum_{i=0}^2 \hat{P}_iT_{i} \; & \; s\in\{1,2,3\} \\
0  \; & \; s\notin\{1,2,3\}
\end{array}\right.
\label{eq:bpsvrd}\end{equation}

\begin{figure}
\centering
\includegraphics{README_files/figure-latex/posbpsv-1.pdf}
\caption{\label{fig:posbpsv}Risk-adjusted R\&D cost for 8 BPSV candidates}
\end{figure}

\subsection{BPSV investigational reserve}\label{bpsv-investigational-reserve}

The cost per dose per year is 2 USD, denoted \(A_1\). Then the cost to maintain the reserve of 100,000 doses is

\begin{equation}
D_{s,y}^{\text{(BP-inv)}} = \left\{\begin{array}{lr}
100000 A_1
\; & \; s\in\{1,2,3\} \;\&\;y>5\\
0  \; & \; s\notin\{1,2,3\}\;\|\;y\leq 5
\end{array}\right.
\label{eq:bpsvinv}\end{equation}

\begin{figure}
\centering
\includegraphics{README_files/figure-latex/bpsvinv-1.pdf}
\caption{\label{fig:bpsvinv}BPSV investigational reserve costs accumulated from year 6 to year 15 with uniformly distributed discount rate.}
\end{figure}

\subsection{SSV capacity reservation}\label{ssv-capacity-reservation}

The cost per dose reservation per year is 0.53 USD, denoted \(A_2\). Reservation sizes, in billions, depend on scenarios, including the \(A_3=0.5\) billion doses reserved for HIC, as follows:

\begin{equation}
M_{R,s} = \left\{\begin{array}{lr}A_3 & s\in\{0, 1, 6, 9, 12\} \\ 
A_3+0.7 & s\in\{2, 4, 7, 10\} \\ 
A_3+2 & s\in\{3, 5, 8, 11\} \end{array}\right.\end{equation}

Then the total cost per year is

\begin{equation}
D_s^{\text{(S-cap)}} =  M_{R,s} A_2
\label{eq:ssvcap}
\end{equation}

The annual costs in billion USD are 0.265, 0.636, and 1.325, respectively.

\begin{figure}
\centering
\includegraphics{README_files/figure-latex/capres-1.pdf}
\caption{\label{fig:capres}Capacity reservation costs accumulated over 15 years with uniformly distributed discount rate.}
\end{figure}

\subsection{Enabling activities}\label{enabling-activities}

Denote the ``Days Mission'' by \(\zeta\), so that \(\zeta\in\{365, 200, 100\}\). Then annual costs, \(E=700\) million, accumulate depending on the year and the mission:

\begin{equation}
D_{s,y}^{\text{(en)}} = \left\{\begin{array}{lr}E & \zeta(s)=200 \;\&\; y\leq 5 \; |\; \zeta(s)=100\; \& \;y\leq 15 \\ 
0 & \zeta(s)=365 \;|\; y > 15 \;|\; \zeta(s)=200 \;\&\; y \;>\; 5  \end{array}\right.
\label{eq:enable}\end{equation}

For our scenarios, we have

\begin{equation}
\zeta(s) = \left\{\begin{array}{lr} 365 & s\in\{0, 1, 2, 3, 4, 5, 12\} \\ 
200 & s\in\{6, 7, 8\} \\ 
100 & s\in\{9, 10, 11\} \end{array}\right.\end{equation}

\begin{figure}
\centering
\includegraphics{README_files/figure-latex/en-1.pdf}
\caption{\label{fig:en}Enabling costs accumulated over 15 years with uniformly distributed discount rate.}
\end{figure}

\section{Response cost equation}\label{response-cost-equation}

{ (BPSV R\&D + SARS-X R\&D + BPSV Procurement + SARS-X Procurement + BPSV Delivery + SARS-X Delivery) / (1 + discount rate) \^{} (year -- 2025) }

\[D_y^{\text{(res)}} = \frac{1}{(1+r)^y}\left(D_s^{\text{(BP-resRD)}} + D_s^{\text{(S-RD)}} + D_s^{\text{(BP-proc)}} + D_{s}^{\text{(S-proc)}} + D_s^{\text{(BP-del)}} + D^{\text{(S-del)}}\right)\]

\begin{itemize}
\tightlist
\item
  \(D_s^{\text{(BP-resRD)}}\) is the R\&D cost of BPSV after an outbreak; see Equation \eqref{eq:bpsvresrd}
\item
  \(D_s^{\text{(S-RD)}}\) is the R\&D cost for SSV; see Equation \eqref{eq:ssvrd}
\item
  \(D_s^{\text{(BP-proc)}}\) is the cost of procuring BPSV; see Equation \eqref{eq:bpsvproc}
\item
  \(D_{s}^{\text{(S-proc)}}\) is the cost of procuring SSV; see Equation \eqref{eq:ssvproc}
\item
  \(D_s^{\text{(BP-del)}}\) is the cost of delivering BPSV; see Equation \eqref{eq:bspvdel}
\item
  \(D^{\text{(S-del)}}\) is the cost of delivering SSV; see Equation \eqref{eq:ssvdel}
\end{itemize}

\subsection{Risk-adjusted R\&D cost per candidate calculation}\label{risk-adjusted-rd-cost-per-candidate-calculation}

{ Sum of the cost of each phase multiplied by the likelihood of phase occurrence (probability of success for previous phases) }

{ Probability of Occurrence (PoO) = 1 * PoS (PhaseN-1) \ldots{} }

{ \$ (Preclin) * PoO (Preclin) + \$ (Ph1) * PoO (Ph1) + \$ (Ph2) * PoO (Ph2) + \$ (Ph3) * PoO (Ph3) + \$ (License) * PoO (License) }

\subsubsection{SSV}\label{ssv}

Trial costs are adjusted for the duration of the trial, which depend on the R\&D investment, denoted \(\zeta\in\{365, 200, 100\}\): \[T_{\zeta,i} = \frac{W_{i;\zeta}^{(S)}}{W_{i;365}^{(S)}}T_i.\] Then the total cost is

\begin{equation}
D_s^{\text{(S-RD)}} = N^{\text{(SSV)}}\left(\sum_{i=0}^3 \hat{P}_iT_{\zeta(s),i} + (1+I) \hat{P}_LL\right)
\label{eq:ssvrd}
\end{equation}

where \(I\) is inflation from 2018 to 2025.

We multiply by the number of candidates, \(N^{\text{(SSV)}}=18\), to get the total cost from the weighted average per candidate.

\includegraphics{README_files/figure-latex/posssv-1.pdf} 365 Days Mission
Min. 1st Qu. Median Mean 3rd Qu. Max.
0.05 0.18 0.24 0.27 0.33 1.99

200 Days Mission
Min. 1st Qu. Median Mean 3rd Qu. Max.
0.01 0.07 0.10 0.13 0.16 1.74

100 Days Mission
Min. 1st Qu. Median Mean 3rd Qu. Max.
0.01 0.04 0.07 0.08 0.10 0.93

\subsubsection{BPSV}\label{bpsv}

\textbf{I have basically assumed the same as SSV except for the numbers given (8 candidates and 18 weeks)}

The BPSV has \(N^{\text{(BPSV)}}=8\) candidates. Those that have passed through Phases 0 to 2 prior to the outbreak go through Phase 3 during the response. The duration is \(Y_3^{(B)}=18\) weeks. Thus we write the BPSV R\&D response cost

\begin{equation}
D_s^{\text{(BP-resRD)}} = \left\{\begin{array}{lr}N^{\text{(BPSV)}}\hat{P}_3\left(\frac{18}{W_{3;365}^{(S)}}\left(\omega T_3^{(n)} + (1-\omega)T_3^{(e)}\right) + (1+I) P_3L\right) \; & \; s\in\{1,2,3\} \\
0  \; & \; s\notin\{1,2,3\}
\end{array}\right.
\label{eq:bpsvresrd}\end{equation}

\includegraphics{README_files/figure-latex/bpsvrd-1.pdf} Min. 1st Qu. Median Mean 3rd Qu. Max.
1 15 29 46 55 907

\subsection{Procurement cost calculation}\label{procurement-cost-calculation}

{ Scenario 1: Annual demand under 6.6B }

{ Annual demand * \$6.29 * 1.14 * 1.2 }

{ Scenario 2: Annual demand over 6.6B }

{ Annual demand * \$18.94 }

\subsubsection{SSV}\label{ssv-1}

If we write annual demand in billions as \(A_{\cdot,s,y}\), then we would have costs, in billion USD, of:

\begin{equation}
D_{s}^{\text{(S-proc)}} = \min\\{A_{SSV,s,y},M_C\\}\cdot S_R\cdot(1+M_p)\cdot(1+M_f)  + \max\\{A_{SSV,s,y}-M_C,0\\}\cdot S_U
\label{eq:ssvproc}
\end{equation}

Here, \(S_R\) is the cost per reserved dose and \(S_U\) the cost per unreserved dose. Reserved doses are marked up by \(M_p\) and \(M_f\).

The total number of doses produced in week \(w\) in scenario \(s\) is \(Z_{T,s,w}\) (see Equation \eqref{eq:supply}). The total in a one-year period is

\[A_{SSV,s,y} = \sum_{w\in y}Z_{T,s,w}.\]

\begin{figure}
\centering
\includegraphics{README_files/figure-latex/costperyear-1.pdf}
\caption{\label{fig:costperyear}SSV procurement cost}
\end{figure}

\subsubsection{BPSV}\label{bpsv-1}

\begin{equation}
D_s^{\text{(BP-proc)}} = \left\{\begin{array}{lr}
A_{BPSV,s}\cdot G\; & \; s\in\{1,2,3\} \\
0  \; & \; s\notin\{1,2,3\}
\end{array}\right.
\label{eq:bpsvproc}\end{equation}

For a world population aged 65 and over of 0.8 billion, and a cost per dose of 4.68 USD, and uptake of 80\%, the procurement cost for BPSV is 3.02 billion USD.

Min. 1st Qu. Median Mean 3rd Qu. Max.
1.26 1.45 1.68 1.70 1.93 2.24

\subsection{Delivery Cost Equation}\label{delivery-cost-equation}

{ WB status demand/0.8 * 0.1 * (0-10\% cost) + WB status demand/0.8 * 0.2 * (11-30\% cost) + WB status demand/0.8 * 0.5 * (30-80\% cost) }

\subsubsection{SSV}\label{ssv-2}

For populations aged 15 and above \(N_i^{(15)}\) in income group \(i\in\{\text{LIC, LMIC, UMIC, HIC}\}\), and delivery cost \(D\):

\begin{equation}
D^{\text{(S-del)}} = \sum_{i}N_i^{(15)}\left(\frac{1}{8}V_{i; 0} + \frac{2}{8}V_{i; 11} + \frac{5}{8}V_{i; 31}\right) 
\label{eq:ssvdel}
\end{equation}

We set

\[V_{LLMIC; j} = \frac{1}{N_{LMIC}^{(15)} + N_{LIC}^{(15)}} \left(N_{LMIC}^{(15)}V_{LMIC; j} + N_{LIC}^{(15)}V_{LIC; j} \right)\]

\begin{figure}
\centering
\includegraphics{README_files/figure-latex/deliverycost-1.pdf}
\caption{\label{fig:deliverycost}SSV delivery cost}
\end{figure}

\subsubsection{BPSV}\label{bpsv-2}

For the BPSV, which goes only to people aged 65 or older, with populations \(N_i^{(65)}\), coverage is reached earlier in the process, so the cost is weighted more heavily towards start up and ramp up:

\begin{equation}
D_s^{\text{(BP-del)}} = 
\left\{\begin{array}{lr}
\sum_{i}D_{\text{BPSV},i}
\; & \; s\in\{1,2,3\} \\
0  \; & \; s\notin\{1,2,3\}
\end{array}\right.
\label{eq:bspvdel}\end{equation}

\begin{equation}
D_{\text{BPSV},i} = 
\left\{\begin{array}{lr}
N_i^{(65)}V_{i; 0}  & N_i^{(65)}\leq \frac{1}{10}N_i^{(15)} \\
\frac{N_i^{(15)}}{10} V_{i; 0} + \left(N_i^{(65)}-\frac{N_i^{(15)}}{10} \right)V_{i; 11}  & \frac{1}{10}N_i^{(15)} \leq N_i^{(65)}\leq \frac{3}{10}N_i^{(15)} \\
\frac{N_i^{(15)}}{10} V_{i; 0} + \frac{2}{10}N_i^{(15)} V_{i; 11} + \left(N_i^{(65)}-\frac{3}{10}N_i^{(15)} \right)V_{i; 31} & N_i^{(65)}> \frac{3}{10} N_i^{(15)}
\end{array}\right.\end{equation}

The logic of this is as follows:

\begin{itemize}
\tightlist
\item
  The increments in cost correspond to numbers of eligible people in the whole population, namely those aged 15 and above.
\item
  If the number of people eligible for the BPSV is less than 10\% of the population aged 15 and over, then all doses cost the ``start up'' amount.
\item
  If the number of people eligible for the BPSV is more than 10\% and less than 30\% of the 15+ population, then cost of the first doses, a number equal to 10\% of the 15+ population, is the ``start up'' amount. All remaining doses cost the ``ramp up'' amount.
\item
  If the number of people eligible for the BPSV is more than 30\% of the 15+ population, then the cost of the first doses, a number equal to 10\% of the 15+ population, is the ``start up'' amount. The cost of the second tranche of doses, a number equal to 20\% of the 15+ population, is the ``ramp up'' amount. All remaining doses cost the ``getting to scale'' amount.
\end{itemize}

\includegraphics{README_files/figure-latex/bpsvdelivery-1.pdf} Min. 1st Qu. Median Mean 3rd Qu. Max.
3.89 5.98 6.94 7.08 8.05 12.09

\begin{longtable}[]{@{}
  >{\raggedright\arraybackslash}p{(\columnwidth - 8\tabcolsep) * \real{0.2000}}
  >{\raggedright\arraybackslash}p{(\columnwidth - 8\tabcolsep) * \real{0.2000}}
  >{\raggedright\arraybackslash}p{(\columnwidth - 8\tabcolsep) * \real{0.2000}}
  >{\raggedright\arraybackslash}p{(\columnwidth - 8\tabcolsep) * \real{0.2000}}
  >{\raggedright\arraybackslash}p{(\columnwidth - 8\tabcolsep) * \real{0.2000}}@{}}
\caption{\label{tab:delcosts} Literature review of global and country-specific delivery costs}\tabularnewline
\toprule\noalign{}
\begin{minipage}[b]{\linewidth}\raggedright
Country
\end{minipage} & \begin{minipage}[b]{\linewidth}\raggedright
Country status
\end{minipage} & \begin{minipage}[b]{\linewidth}\raggedright
Study type
\end{minipage} & \begin{minipage}[b]{\linewidth}\raggedright
Financial Cost per dose (USD)
\end{minipage} & \begin{minipage}[b]{\linewidth}\raggedright
Source
\end{minipage} \\
\midrule\noalign{}
\endfirsthead
\toprule\noalign{}
\begin{minipage}[b]{\linewidth}\raggedright
Country
\end{minipage} & \begin{minipage}[b]{\linewidth}\raggedright
Country status
\end{minipage} & \begin{minipage}[b]{\linewidth}\raggedright
Study type
\end{minipage} & \begin{minipage}[b]{\linewidth}\raggedright
Financial Cost per dose (USD)
\end{minipage} & \begin{minipage}[b]{\linewidth}\raggedright
Source
\end{minipage} \\
\midrule\noalign{}
\endhead
\bottomrule\noalign{}
\endlastfoot
WHO, Gavi, and UNICEF AMC Estimate & AMC & Top down & 1.66 & \citet{Griffiths2021} \\
UNICEF Global Estimate & All & Model & 0.73 & \citet{Oyatoye2023} \\
DRC & LIC & Bottom up & 1.91 & \citet{Moi2024} \\
Malawi & LIC & Bottom up & 4.55 & \citet{Ruisch2025} \\
Mozambique & LIC & Bottom up & 0.5 & \citet{Namalela2025} \\
Uganda & LIC & Bottom up & 0.79 & \citet{Tumusiime2024} \\
Bangladesh & LMIC & Bottom up & 0.29 & \citet{Yesmin2024} \\
Cote d'Ivoire & LMIC & Bottom up & 0.67 & \citet{Vaughan2023} \\
Nigeria & LMIC & Bottom up & 0.84 & \citet{Noh2024} \\
Philippines & LMIC & Bottom up & 2.16 & \citet{Banks2023} \\
Vietnam & LMIC & Bottom up & 1.73 & \citet{Nguyen2024} \\
Ghana & LMIC & CVIC tool & 2.2--2.3 & \citet{Nonvignon2022} \\
Lao PDR & LMIC & CVIC tool & 0.79--0.81 & \citet{Yeung2023} \\
Kenya & LMIC & Top down & 3.29--4.28 & \citet{Orangi2022} \\
Botswana & UMIC & Mixed & 19 & \citet{Vaughan2025} \\
South Africa & UMIC & Top down & 3.84 & \citet{Edoka2024} \\
\end{longtable}

\section{SSV delivery}\label{ssv-delivery}

\begin{longtable}[]{@{}
  >{\raggedright\arraybackslash}p{(\columnwidth - 6\tabcolsep) * \real{0.2500}}
  >{\raggedright\arraybackslash}p{(\columnwidth - 6\tabcolsep) * \real{0.2500}}
  >{\raggedright\arraybackslash}p{(\columnwidth - 6\tabcolsep) * \real{0.2500}}
  >{\raggedright\arraybackslash}p{(\columnwidth - 6\tabcolsep) * \real{0.2500}}@{}}
\caption{Manufacturing response timeline assumptions}\tabularnewline
\toprule\noalign{}
\begin{minipage}[b]{\linewidth}\raggedright
Category
\end{minipage} & \begin{minipage}[b]{\linewidth}\raggedright
Reserved capacity
\end{minipage} & \begin{minipage}[b]{\linewidth}\raggedright
Private response (existing capacity)
\end{minipage} & \begin{minipage}[b]{\linewidth}\raggedright
Private response (built capacity)
\end{minipage} \\
\midrule\noalign{}
\endfirsthead
\toprule\noalign{}
\begin{minipage}[b]{\linewidth}\raggedright
Category
\end{minipage} & \begin{minipage}[b]{\linewidth}\raggedright
Reserved capacity
\end{minipage} & \begin{minipage}[b]{\linewidth}\raggedright
Private response (existing capacity)
\end{minipage} & \begin{minipage}[b]{\linewidth}\raggedright
Private response (built capacity)
\end{minipage} \\
\midrule\noalign{}
\endhead
\bottomrule\noalign{}
\endlastfoot
Annual manufacturing volume & By scenario (0.5--2.5B) & 2.5B minus reserved volume & 6B \\
Facility transition start & 7 weeks before vaccine approval & 7 weeks before vaccine approval & 7 weeks before vaccine approval \\
Weeks to initial manufacturing & 12 & 30 & 48 \\
Scale-up weeks to full capacity & 10 & 16 & 16 \\
\end{longtable}

\begin{longtable}[]{@{}
  >{\raggedright\arraybackslash}p{(\columnwidth - 6\tabcolsep) * \real{0.2500}}
  >{\raggedright\arraybackslash}p{(\columnwidth - 6\tabcolsep) * \real{0.2500}}
  >{\raggedright\arraybackslash}p{(\columnwidth - 6\tabcolsep) * \real{0.2500}}
  >{\raggedright\arraybackslash}p{(\columnwidth - 6\tabcolsep) * \real{0.2500}}@{}}
\caption{Vaccine Production Timeline}\tabularnewline
\toprule\noalign{}
\begin{minipage}[b]{\linewidth}\raggedright
Weeks from transition start
\end{minipage} & \begin{minipage}[b]{\linewidth}\raggedright
Reserved Capacity (\%)
\end{minipage} & \begin{minipage}[b]{\linewidth}\raggedright
Private Capacity (Existing; \%)
\end{minipage} & \begin{minipage}[b]{\linewidth}\raggedright
Private Capacity (Response; \%)
\end{minipage} \\
\midrule\noalign{}
\endfirsthead
\toprule\noalign{}
\begin{minipage}[b]{\linewidth}\raggedright
Weeks from transition start
\end{minipage} & \begin{minipage}[b]{\linewidth}\raggedright
Reserved Capacity (\%)
\end{minipage} & \begin{minipage}[b]{\linewidth}\raggedright
Private Capacity (Existing; \%)
\end{minipage} & \begin{minipage}[b]{\linewidth}\raggedright
Private Capacity (Response; \%)
\end{minipage} \\
\midrule\noalign{}
\endhead
\bottomrule\noalign{}
\endlastfoot
0--11 & & & \\
12--21 & Scaling from 0-100 & & \\
22--29 & 100 & & \\
30--45 & 100 & Scaling from 0-100 & \\
46--47 & 100 & 100 & \\
48--63 & 100 & 100 & Scaling from 0-100 \\
64+ & 100 & 100 & 100 \\
\end{longtable}

\subsection{Timing}\label{timing}

Facility transition occurs \(F=7\) weeks before vaccine approval, which in turn depends on R\&D investments. We have three levels in our scenarios, corresponding to a 100 Days Mission, 200 days, and 365 days. The total weeks taken for vaccine approval can be written as follows:

\[W_{j}^{(S)} = \sum_{i=0}^3 W_{i;j}^{(S)}\]

for \(j\in\\{365, 200, 100\\}\). These work out as 68, 27, and 13 weeks, respectively. Thus ``week 0'' for manufacturing occurs 61, 20, and 6 weeks, respectively, after the new pathogen has been sequenced. We denote this variable \(w_s^{(0)}\).

\subsection{Production}\label{production}

The total global manufacturing volume is \(M_G=15\) billion doses. The amount that is reserved, in billion doses, including the HIC-specific reservation of \(A_3=0.5\) billion doses, depends on the scenarios as follows:

\begin{equation}
M_{R,s} = \left\{\begin{array}{lr}A_3 & s\in\{0, 1, 6, 9, 12\} \\ 
A_3 + 0.7 & s\in\{2, 4, 7, 10\} \\ 
A_3 + 2 & s\in\{3, 5, 8, 11\} \end{array}\right.\end{equation}

where \(s=0\) denotes the BAU scenario. By definition, \(M_{E,s} = M_C - M_{R,s}\), and \(M_B=M_G-M_C\).

Then the number of doses, in billions, that are made from capacity \(x\in \\{R, E, B\\}\) in week \(w\) of scenario \(s\) is:

\begin{equation}
Z_{x,s,w} = \left\{\begin{array}{lr}0 & w-w_s^{(0)} < I_x \\ 
\frac{1}{52}\frac{w-w_s^{(0)}-I_x+1}{C_x}M_{x,s} & w-w_s^{(0)}\in[I_x, I_x+C_x) \\ 
\frac{1}{52}M_{x,s}  & w-w_s^{(0)}\geq I_x+C_x
\end{array}\right.\end{equation}

where \(I_R=12\) is the number of weeks to initial manufacturing for reserved capacity, \(C_R=10\) is the number of weeks to scale up to full capacity; \(I_E=30\) is the number of weeks to initial manufacturing for existing and unreserved capacity, \(C_E=16\) is the number of weeks to scale up to full capacity; \(I_B=48\) is the number of weeks to initial manufacturing for built and unreserved capacity, \(C_B=16\) is the number of weeks to scale up to full capacity.

Then the total number of doses produced in week \(w\) is

\begin{equation}
Z_{T,s,w} = Z_{R,s,w}+Z_{E,s,w}+Z_{B,s,w}.
\label{eq:supply}
\end{equation}

\begin{figure}
\centering
\includegraphics{README_files/figure-latex/supply-1.pdf}
\caption{\label{fig:supply}Doses made available from manufacturing per scenario. Weeks are in reference to the sequencing of the pathogen.}
\end{figure}

In Figure \ref{fig:supply}, the following scenarios have identical supply (because they have the same capacity reservations and R\&D investments): BAU \& S01 \& S12; S02 \& S04; and S03 \& S05.

\subsection{Allocation}\label{allocation}

Denote the weekly allocated doses at week \(w\) from capacity \(x\) to income level \(k_{s,x,i,w}\), and the cumulative number \(K_{s,i,w}\), such that \[K_{s,i,w} = \sum_{x\in\\{R,E,B\\}}\sum_{j=0}^w k_{s,x,i,j}.\] We write \(X_i = 2\cdot \lambda\cdot N_i^{(15)}\) as the maximum demand for income group \(i\), representing two doses each for \(\lambda=80\)\% of the population.

\begin{equation}
k_{s,R,i,w} = \left\{ \begin{array}{lr}
Z_{R,s,w}             & K_{s,\text{HIC},w} < A_3 \;\&\; i=\text{HIC} \\
0                     & K_{s,\text{HIC},w} < A_3 \;\&\; i\neq\text{HIC} \\
\frac{N_{i}}{N_{HIC}+N_{UMIC}+N_{LLMIC}}Z_{R,s,w} & A_3 < K_{s,\text{HIC},w} < X_{\text{HIC}} \\
\frac{N_{i}}{N_{UMIC}+N_{LLMIC}}Z_{R,s,w} & K_{s,\text{HIC},w} \geq X_{\text{HIC}} \;\&\;  K_{s,\text{UMIC},w} < X_{\text{HIC}} \;\&\; i\neq\text{HIC}\\
Z_{R,s,w}             & K_{s,\text{UMIC},w} \geq X_{\text{UMIC}} \;\&\; i=\text{LLMIC}
\end{array}\right.\end{equation}

The logic of this reads as follows:

\begin{itemize}
\tightlist
\item
  The first \(A_3=0.5\) billion doses from reserved capacity go exclusively to HIC
\item
  None go to UMIC and LLMIC
\item
  When HIC coverage is between 500 million and its total demand, reserved capacity doses are allocated according to population
\item
  Once HIC reach their total demand, doses from reserved capacity are split proportional to population between UMIC and LLMIC
\item
  Once UMIC reach their total demand, all doses from reserved capacity go to LLMIC
\end{itemize}

For \(x\in\\{E,B\\}\),

\begin{equation}
k_{s,x,i,w} = \left\{ \begin{array}{lr}
Z_{x,s,w}            & K_{s,\text{HIC},w} < X_{\text{HIC}} \;\&\; i=\text{HIC} \\
0                     & K_{s,\text{HIC},w} < X_{\text{HIC}} \;\&\; i\neq\text{HIC} \\
Z_{x,s,w}            & K_{s,\text{HIC},w} \geq X_{\text{HIC}} \;\&\; K_{s,\text{UMIC},w} < X_{\text{UMIC}} \;\&\; i=\text{UMIC} \\
0                     & K_{s,\text{HIC},w} \geq X_{\text{HIC}} \;\&\; K_{s,\text{UMIC},w} < X_{\text{UMIC}} \;\&\; i\neq\text{UMIC} \\
Z_{x,s,w}            & K_{s,\text{UMIC},w} \geq X_{\text{UMIC}} \;\&\; i=\text{LLMIC} \\
0                     & K_{s,\text{UMIC},w} \geq X_{\text{UMIC}} \;\&\; i\neq\text{LLMIC} 
\end{array}\right.\end{equation}

The logic of this reads as follows:

\begin{itemize}
\tightlist
\item
  Until HIC demand is reached, all doses from unreserved capacity go to HIC
\item
  None go to UMIC and LLMIC
\item
  Once HIC demand has been met and until UMIC demand is reached, all doses from unreserved capacity go to UMIC
\item
  None go to HIC and LLMIC
\item
  Once HIC and UMIC demand have been met, all remaining doses from unreserved capacity go to LLMIC
\item
  None go to UMIC and HIC
\end{itemize}

\begin{figure}
\centering
\includegraphics{README_files/figure-latex/procurement-1.pdf}
\caption{\label{fig:procurement}Doses procured by country income level}
\end{figure}

\subsection{Delivery}\label{delivery}

\begin{figure}
\centering
\includegraphics{README_files/figure-latex/scendelivery-1.pdf}
\caption{\label{fig:scendelivery}Cumulative vaccine coverage (second SSV dose) by country income level}
\end{figure}

\section{BPSV delivery}\label{bpsv-delivery}

\subsection{Timing}\label{timing-1}

The duration of the Phase three trial is 18 weeks. \textbf{The time to manufacturing transition is 12 weeks, and the time to manufacturing scale-up 10 weeks; these are the same as the reserved-capacity times for SSV. Are these reservations included in the preparedness cost?}

\section{Parameter samples}\label{parameter-samples}

\includegraphics{README_files/figure-latex/parsamples-1.pdf} \includegraphics{README_files/figure-latex/parsamples-2.pdf} \includegraphics{README_files/figure-latex/parsamples-3.pdf} \includegraphics{README_files/figure-latex/parsamples-4.pdf} \includegraphics{README_files/figure-latex/parsamples-5.pdf} \includegraphics{README_files/figure-latex/parsamples-6.pdf} \includegraphics{README_files/figure-latex/parsamples-7.pdf} \includegraphics{README_files/figure-latex/parsamples-8.pdf} \includegraphics{README_files/figure-latex/parsamples-9.pdf} \includegraphics{README_files/figure-latex/parsamples-10.pdf} \includegraphics{README_files/figure-latex/parsamples-11.pdf} \includegraphics{README_files/figure-latex/parsamples-12.pdf} \includegraphics{README_files/figure-latex/parsamples-13.pdf} \includegraphics{README_files/figure-latex/parsamples-14.pdf} \includegraphics{README_files/figure-latex/parsamples-15.pdf} \includegraphics{README_files/figure-latex/parsamples-16.pdf} \includegraphics{README_files/figure-latex/parsamples-17.pdf} \includegraphics{README_files/figure-latex/parsamples-18.pdf} \includegraphics{README_files/figure-latex/parsamples-19.pdf} \includegraphics{README_files/figure-latex/parsamples-20.pdf} \includegraphics{README_files/figure-latex/parsamples-21.pdf} \includegraphics{README_files/figure-latex/parsamples-22.pdf} \includegraphics{README_files/figure-latex/parsamples-23.pdf} \includegraphics{README_files/figure-latex/parsamples-24.pdf} \includegraphics{README_files/figure-latex/parsamples-25.pdf} \includegraphics{README_files/figure-latex/parsamples-26.pdf} \includegraphics{README_files/figure-latex/parsamples-27.pdf}

\section{Attributions / Authors}\label{attributions-authors}

\renewcommand\refname{References}
  \bibliography{../../epi.bib}

\end{document}
