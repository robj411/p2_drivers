% Options for packages loaded elsewhere
\PassOptionsToPackage{unicode}{hyperref}
\PassOptionsToPackage{hyphens}{url}
\documentclass[
]{article}
\usepackage{xcolor}
\usepackage[margin=1in]{geometry}
\usepackage{amsmath,amssymb}
\setcounter{secnumdepth}{5}
\usepackage{iftex}
\ifPDFTeX
  \usepackage[T1]{fontenc}
  \usepackage[utf8]{inputenc}
  \usepackage{textcomp} % provide euro and other symbols
\else % if luatex or xetex
  \usepackage{unicode-math} % this also loads fontspec
  \defaultfontfeatures{Scale=MatchLowercase}
  \defaultfontfeatures[\rmfamily]{Ligatures=TeX,Scale=1}
\fi
\usepackage{lmodern}
\ifPDFTeX\else
  % xetex/luatex font selection
\fi
% Use upquote if available, for straight quotes in verbatim environments
\IfFileExists{upquote.sty}{\usepackage{upquote}}{}
\IfFileExists{microtype.sty}{% use microtype if available
  \usepackage[]{microtype}
  \UseMicrotypeSet[protrusion]{basicmath} % disable protrusion for tt fonts
}{}
\makeatletter
\@ifundefined{KOMAClassName}{% if non-KOMA class
  \IfFileExists{parskip.sty}{%
    \usepackage{parskip}
  }{% else
    \setlength{\parindent}{0pt}
    \setlength{\parskip}{6pt plus 2pt minus 1pt}}
}{% if KOMA class
  \KOMAoptions{parskip=half}}
\makeatother
\usepackage{longtable,booktabs,array}
\usepackage{calc} % for calculating minipage widths
% Correct order of tables after \paragraph or \subparagraph
\usepackage{etoolbox}
\makeatletter
\patchcmd\longtable{\par}{\if@noskipsec\mbox{}\fi\par}{}{}
\makeatother
% Allow footnotes in longtable head/foot
\IfFileExists{footnotehyper.sty}{\usepackage{footnotehyper}}{\usepackage{footnote}}
\makesavenoteenv{longtable}
\usepackage{graphicx}
\makeatletter
\newsavebox\pandoc@box
\newcommand*\pandocbounded[1]{% scales image to fit in text height/width
  \sbox\pandoc@box{#1}%
  \Gscale@div\@tempa{\textheight}{\dimexpr\ht\pandoc@box+\dp\pandoc@box\relax}%
  \Gscale@div\@tempb{\linewidth}{\wd\pandoc@box}%
  \ifdim\@tempb\p@<\@tempa\p@\let\@tempa\@tempb\fi% select the smaller of both
  \ifdim\@tempa\p@<\p@\scalebox{\@tempa}{\usebox\pandoc@box}%
  \else\usebox{\pandoc@box}%
  \fi%
}
% Set default figure placement to htbp
\def\fps@figure{htbp}
\makeatother
\setlength{\emergencystretch}{3em} % prevent overfull lines
\providecommand{\tightlist}{%
  \setlength{\itemsep}{0pt}\setlength{\parskip}{0pt}}
\usepackage[]{natbib}
\bibliographystyle{plainnat}
\usepackage{float}
\usepackage{booktabs}
\usepackage{longtable}
\usepackage{array}
\usepackage{multirow}
\usepackage{wrapfig}
\usepackage{colortbl}
\usepackage{pdflscape}
\usepackage{tabu}
\usepackage{threeparttable}
\usepackage{threeparttablex}
\usepackage[normalem]{ulem}
\usepackage{makecell}
\usepackage{xcolor}
\usepackage{bookmark}
\IfFileExists{xurl.sty}{\usepackage{xurl}}{} % add URL line breaks if available
\urlstyle{same}
\hypersetup{
  pdftitle={The Costing Model},
  hidelinks,
  pdfcreator={LaTeX via pandoc}}

\title{The Costing Model}
\author{}
\date{\vspace{-2.5em}}

\begin{document}
\maketitle

This document describes the costing model that is used in the CEPI application.

\section{Parameters}\label{parameters}

\begin{longtable}[]{@{}
  >{\centering\arraybackslash}p{(\linewidth - 8\tabcolsep) * \real{0.1714}}
  >{\centering\arraybackslash}p{(\linewidth - 8\tabcolsep) * \real{0.2357}}
  >{\centering\arraybackslash}p{(\linewidth - 8\tabcolsep) * \real{0.1357}}
  >{\centering\arraybackslash}p{(\linewidth - 8\tabcolsep) * \real{0.2286}}
  >{\centering\arraybackslash}p{(\linewidth - 8\tabcolsep) * \real{0.2286}}@{}}
\caption{Notation and parametric assumptions for inputs to the costing model. Parameters are used as follows: uniform distributions go from Parameter 1 to Parameter 2. Triangular distributions go from Parameter 1 to Parameter 3 with a peak at Parameter 2. Multinomial distributions have equally probable values listed individually. Exponential distributions have as a mean Parameter 1. Inverse Gaussian distributions have as a mean Parameter 1, and as a shape Parameter 2. Log normal distributions have as a mean Parameter 1, and as a standard deviation Parameter 2. PearsonV distributions have shape Parameter 1, scale Parameter 2, and location 0. PearsonVI distributions have shape Parameters 1 and 2, scale Parameter 3, and location 0. Where given, distributions are truncated at bounds.}\tabularnewline
\toprule\noalign{}
\begin{minipage}[b]{\linewidth}\centering
Math notation
\end{minipage} & \begin{minipage}[b]{\linewidth}\centering
Description
\end{minipage} & \begin{minipage}[b]{\linewidth}\centering
Distribution
\end{minipage} & \begin{minipage}[b]{\linewidth}\centering
Parameters
\end{minipage} & \begin{minipage}[b]{\linewidth}\centering
Source
\end{minipage} \\
\midrule\noalign{}
\endfirsthead
\toprule\noalign{}
\begin{minipage}[b]{\linewidth}\centering
Math notation
\end{minipage} & \begin{minipage}[b]{\linewidth}\centering
Description
\end{minipage} & \begin{minipage}[b]{\linewidth}\centering
Distribution
\end{minipage} & \begin{minipage}[b]{\linewidth}\centering
Parameters
\end{minipage} & \begin{minipage}[b]{\linewidth}\centering
Source
\end{minipage} \\
\midrule\noalign{}
\endhead
\bottomrule\noalign{}
\endlastfoot
\(N^{\text{(BPSV)}}\) & Number of BPSV candidates & Constant & 14 & \\
\(N^{\text{(BPSV-1)}}\) & Number of BPSV candidates
starting at phase 1 & Constant & 1 & \\
\(P_0^{\text{(BPSV)}}\) & Probability of success;
preclinical & Multinomial & 0.40, 0.41, 0.41, 0.42, 0.48,
0.57 & \citet{Gouglas2018} \\
\(P_1^{\text{(BPSV)}}\) & Probability of success; Phase
I & Multinomial & 0.33, 0.40, 0.50, 0.68, 0.70,
0.72, 0.74, 0.77, 0.81, 0.90 & \citet{Gouglas2018} \\
\(P_2^{\text{(BPSV)}}\) & Probability of success; Phase
II & Multinomial & 0.22, 0.31, 0.33, 0.43, 0.46,
0.54, 0.58, 0.58, 0.74, 0.79 & \citet{Gouglas2018} \\
\(Y_0^{(B)}\) & BPSV preclinical duration;
years & Multinomial & 1, 2 & \citet{CEPI2022} \\
\(Y_1^{(B)}\) & BPSV Phase I duration; years & Multinomial & 1, 2 & \citet{CEPI2022} \\
\(Y_2^{(B)}\) & BPSV Phase II duration; years & Constant & 2 & \citet{CEPI2022} \\
\(Y_3^{(B)}\) & BPSV Phase III duration; years & Multinomial & 3, 4 & \citet{CEPI2022} \\
\(W_3^{(B)}\) & BPSV response Phase III
duration; weeks & Constant & 18 & \\
\(Q^{\text{(SSV)}}\) & Probability of N or more SSV
successes & Constant & 0.9 & Model choice \\
\(n^{\text{(SSV)}}\) & Number of SSV successes & Constant & 5 & Model choice \\
\(X_0\) & COVID-19 candidates failed at
preclinical & Constant & 33 & \citet{LinksbridgeSPC2025} \\
\(X_1\) & COVID-19 candidates failed at
Phase 1 & Constant & 20 & \citet{LinksbridgeSPC2025} \\
\(X_2\) & COVID-19 candidates failed at
Phase 2 & Constant & 8 & \citet{LinksbridgeSPC2025} \\
\(X_3\) & COVID-19 candidates failed at
Phase 3 & Constant & 8 & \citet{LinksbridgeSPC2025} \\
\(X_4\) & COVID-19 candidates successful & Constant & 27 & \citet{LinksbridgeSPC2025} \\
\(Y^{(200)}\) & Years of R\&D to 200-day
readiness & Constant & 5 & \\
\(Y^{(100)}\) & Years of R\&D to 100-day
readiness & Constant & 15 & \\
\(E\) & Enabling activities; million
USD per year & Constant & 700 & \citet{CEPI2021} \\
\(W_{0; 365}^{(S)}\) & SSV preclinical duration
(365); weeks & Constant & 14 & \\
\(W_{0; 200}^{(S)}\) & SSV preclinical duration (200
days); weeks & Constant & 5 & \\
\(W_{0; 100}^{(S)}\) & SSV preclinical duration (100
days); weeks & Constant & 5 & \\
\(W_{1; 365}^{(S)}\) & SSV phase I duration (365);
weeks & Constant & 0 & \\
\(W_{1; 200}^{(S)}\) & SSV phase I duration (200
days); weeks & Constant & 0 & \\
\(W_{1; 100}^{(S)}\) & SSV phase I duration (100
days); weeks & Constant & 0 & \\
\(W_{2; 365}^{(S)}\) & SSV phase II duration (365);
weeks & Constant & 19 & \\
\(W_{2; 200}^{(S)}\) & SSV phase II duration (200
days); weeks & Constant & 7 & \\
\(W_{2; 100}^{(S)}\) & SSV phase II duration (100
days); weeks & Constant & 0 & \\
\(W_{3; 365}^{(S)}\) & SSV phase III duration (365);
weeks & Constant & 16 & \\
\(W_{3; 200}^{(S)}\) & SSV phase III duration (200
days); weeks & Constant & 15 & \\
\(W_{3; 100}^{(S)}\) & SSV phase III duration (100
days); weeks & Constant & 8 & \\
\(T_0^{(e)}\) & Cost, preclinical, experienced
manufacturer; USD & Exponential & 24213683 & \citet{Gouglas2018} \\
\(T_0^{(n)}\) & Cost, preclinical,
inexperienced manufacturer;
USD & Inverse Gaussian & 7882792, 13455907 & \citet{Gouglas2018} \\
\(T_1^{(e)}\) & Cost, Phase I, experienced
manufacturer; USD & Inverse Gaussian & 15339198, 8076755 & \citet{Gouglas2018} \\
\(T_1^{(n)}\) & Cost, Phase I, inexperienced
manufacturer; USD & PearsonV & 2.2774, 9799081 & \citet{Gouglas2018} \\
\(T_2^{(e)}\) & Cost, Phase II, experienced
manufacturer; USD & Log normal & 28297339, 24061641 & \citet{Gouglas2018} \\
\(T_2^{(n)}\) & Cost, Phase II, inexperienced
manufacturer; USD & Inverse Gaussian & 17124622, 35918793 & \citet{Gouglas2018} \\
\(T_3^{(e)}\) & Cost, Phase III, experienced
manufacturer; USD & PearsonV & 1.3147, 51397313 & \citet{Gouglas2018} \\
\(T_4\) & Licensure cost, 2018; USD & Constant & 287750 & \citet{Gouglas2018} \\
\(I\) & Inflation (2018 t0 2025) & Constant & 0.28000000000000003 & \citet{U.S.BLS2025} \\
\(\omega\) & Share of manufacturers that
are inexperienced & Constant & 0.92307692307692313 & See Table \ref{tab:inex} \\
\(r\) & Discount rate & Uniform & 0.02, 0.06 & \citet{Glennerster2023} \\
\(A_4\) & Size of BPSV investigational
reserve, doses & Constant & 100000 & Model choice \\
\(A_1\) & Annual BPSV reservation cost,
USD per dose & Constant & 1.0121363333333333E-2 & \\
\(A_5\) & BPSV reserve upfront cost, USD
per dose & Constant & 0.115 & \\
\(Y_{rep}\) & Years after which BPSV doses
are to be replaced & Constant & 3 & \\
\(A_2\) & Advanced capacity reservation
fee; USD per dose per year & Constant & 0.53169230769230769 & \citet{Pfizer2023} \\
\(A_3\) & Reserved capacity for HIC,
billions & Constant & 0.5 & \\
\(S_U\) & SSV procurement price,
reactive capacity; USD per
dose & Constant & 18.9392 & \citet{LinksbridgeSPC2025} \\
\(G\) & Drug substance cost; USD per
dose & Constant & 4.68 & \citet{Kazaz2021} \\
\(M_p\) & Profit margin & Constant & 0.2 & \citet{Kazaz2021} \\
\(M_f\) & Fill/finish cost & Constant & 0.13980000000000001 & \citet{Kazaz2021} \\
\(M_t\) & Cost to transport product & Constant & 0.12 & \citet{Kazaz2021} \\
\(M_G\) & Global annual manufacturing
volume; billion doses & Constant & 15 & \citet{LinksbridgeSPC2025} \\
\(M_C\) & Current annual manufacturing
volume; billion doses & Constant & 9 & \citet{LinksbridgeSPC2025} \\
\(\lambda\) & Final vaccine coverage,
proportion of population & Constant & 0.8 & Model choice \\
\(\delta\) & Fraction of BPSV expected to
go to waste & Constant & 0.31425320000000001 & Model choice \\
\(N^{\text{(boost)}}\) & Number of boosters given, one
per year & Constant & 2 & Model choice \\
\(I_0\) & Facility transition start;
weeks before vaccine approval & Constant & 7 & \\
\(I_R\) & Weeks to initial
manufacturing, reserved
infrastructure & Constant & 12 & \citet{VaccinesEurope2023} \\
\(I_{E,0}\) & Weeks to initial manufacturing
when there's no BPSV, existing
and unreserved infrastructure & Constant & 30 & \citet{VaccinesEurope2023} \\
\(I_{E,1}\) & Weeks to initial manufacturing
when there's BPSV, existing
and unreserved infrastructure & Constant & 12 & \citet{VaccinesEurope2023} \\
\(I_B\) & Weeks to initial
manufacturing, built and
unreserved infrastructure & Constant & 48 & \\
\(C_R\) & Weeks to scale up to full
capacity, reserved
infrastructure & Constant & 10 & \citet{VaccinesEurope2023} \\
\(C_E\) & Weeks to scale up to full
capacity, existing and
unreserved infrastructure & Constant & 16 & \\
\(C_B\) & Weeks to scale up to full
capacity, built and unreserved
infrastructure & Constant & 16 & \\
\(V_{L; 0}\) & Cost of vaccine delivery at
start up (0--10\%) in LIC;
USD per dose & Triangular & 1, 1.5, 2 & See Table \ref{tab:delcosts} \\
\(V_{L; 11}\) & Cost of vaccine delivery
during ramp up (11--30\%) in
LIC; USD per dose & Triangular & 0.75, 1, 1.5 & See Table \ref{tab:delcosts} \\
\(V_{L; 31}\) & Cost of vaccine delivery
getting to scale (31\% and
over) in LIC; USD per dose & Triangular & 1, 2, 4 & See Table \ref{tab:delcosts} \\
\(V_{LM; 0}\) & Cost of vaccine delivery at
start up (0--10\%) in LMIC;
USD per dose & Triangular & 3, 4.5, 6 & See Table \ref{tab:delcosts} \\
\(V_{LM; 11}\) & Cost of vaccine delivery
during ramp up (11--30\%) in
LMIC; USD per dose & Triangular & 2.25, 3, 4.5 & See Table \ref{tab:delcosts} \\
\(V_{LM; 31}\) & Cost of vaccine delivery
getting to scale (31\% and
over) in LMIC; USD per dose & Triangular & 1.5, 2, 2.5 & See Table \ref{tab:delcosts} \\
\(V_{UM; 0}\) & Cost of vaccine delivery at
start up (0--10\%) in UMIC;
USD per dose & Triangular & 6, 9, 12 & See Table \ref{tab:delcosts} \\
\(V_{UM; 11}\) & Cost of vaccine delivery
during ramp up (11--30\%) in
UMIC; USD per dose & Triangular & 4.5, 6, 9 & See Table \ref{tab:delcosts} \\
\(V_{UM; 31}\) & Cost of vaccine delivery
getting to scale (31\% and
over) in UMIC; USD per dose & Triangular & 3, 4, 5 & See Table \ref{tab:delcosts} \\
\(V_{H; 0}\) & Cost of vaccine delivery at
start up (0--10\%) in HIC;
USD per dose & Triangular & 30, 40, 75 & See Table \ref{tab:delcosts} \\
\(V_{H; 11}\) & Cost of vaccine delivery
during ramp up (11--30\%) in
HIC; USD per dose & Triangular & 30, 40, 75 & See Table \ref{tab:delcosts} \\
\(V_{H; 31}\) & Cost of vaccine delivery
getting to scale (31\% and
over) in HIC; USD per dose & Triangular & 30, 40, 75 & See Table \ref{tab:delcosts} \\
\(N_{HIC}^{(0)}\) & Population, HIC & Constant & 1260028362 & \\
\(N_{UMIC}^{(0)}\) & Population, UMIC & Constant & 2854556263.5 & \\
\(N_{LMIC}^{(0)}\) & Population, LMIC & Constant & 3314048516 & \\
\(N_{LIC}^{(0)}\) & Population, LIC & Constant & 762656294.5 & \\
\(N_{HIC}^{(15)}\) & Population aged 15 and older,
HIC & Constant & 1064531991.5 & \\
\(N_{UMIC}^{(15)}\) & Population aged 15 and older,
UMIC & Constant & 2308984518 & \\
\(N_{LMIC}^{(15)}\) & Population aged 15 and older,
LMIC & Constant & 2363976954.5 & \\
\(N_{LIC}^{(15)}\) & Population aged 15 and older,
LIC & Constant & 450976596.5 & \\
\(N_{HIC}^{(65)}\) & Population aged 65 and older,
HIC & Constant & 256715334 & \\
\(N_{UMIC}^{(65)}\) & Population aged 65 and older,
UMIC & Constant & 359824402.5 & \\
\(N_{LMIC}^{(65)}\) & Population aged 65 and older,
LMIC & Constant & 215830985.5 & \\
\(N_{LIC}^{(65)}\) & Population aged 65 and older,
LIC & Constant & 24812768 & \\
\end{longtable}

\section{Preparedness cost equation}\label{preparedness-cost-equation}

We can write the total preparedness cost for scenario \(s\) in year \(y\) as

\[
D_y^{\text{(prep)}} = \frac{1}{(1+r)^y}\left(D_s^{\text{(BP-adRD)}} + D_{s,y}^{\text{(BP-man)}} + D_{s,y}^{\text{(BP-inv)}} + D_s^{\text{(S-cap)}} + D_{s,y}^{\text{(en)}}\right)
\]

where:

\begin{itemize}
\tightlist
\item
  \(D_s^{\text{(BP-adRD)}}\) is the R\&D cost of BPSV prior to an outbreak; see Equation \eqref{eq:bpsvrd}
\item
  \(D_{s,y}^{\text{(BP-man)}}\) is the upfront cost of maintaining an investigational reserve of 100,000 BPSV doses; see Equation \eqref{eq:bpsvadman}
\item
  \(D_{s,y}^{\text{(BP-inv)}}\) is the annual cost of maintaining an investigational reserve of 100,000 BPSV doses; see Equation \eqref{eq:bpsvinv}
\item
  \(D_s^{\text{(S-cap)}}\) is the cost of reserved capacity for SSV; see Equation \eqref{eq:ssvcap}
\item
  \(D_{s,y}^{\\text{(en)}}\) is the annual cost of enabling activities; see Equation \eqref{eq:enable}.
\end{itemize}

\subsection{BPSV advanced R\&D}\label{bpsv-advanced-rd}

\textbf{These values match the spreadsheet results}

Advanced R\&D for BPSVs consists of Phases 0 to II, for which we add up costs that depend on (a) the number of candidates, (b) the cost per phase for experienced and inexperienced developers, and (c) the probability of success in each phase. The final cost therefore reflects the number of candidates that progressed through the developmental pipeline.

Probabilities of success for preclinical and Phase I are \(P_0^{\text{(BPSV)}}\) and \(P_1^{\text{(BPSV)}}\). Then probabilities of phase occurrence are:

\[
\hat{P}_i^{(0)} = \begin{cases}1 & i=0 \\ 
\prod_{j=0}^{i-1}P_j^{\text{(BPSV)}} & i\in\lbrace 1,2\rbrace 
\end{cases}
\]

For \(N^{\text{(BPSV-1)}} = 1\) candidate(s), which have already been through the preclinical phase, we have

\[
\hat{P}_i^{(1)} = \begin{cases}1 & i=1 \\ 
\prod_{j=1}^{i-1}P_j^{\text{(BPSV)}} & i>1 
\end{cases}
\]

The cost of each phase is \(T_i\), a weighted average of experienced and inexperienced manufacturers. Assuming that \(N^{\text{(BPSV)}}-N^{\text{(BPSV-1)}}=\) candidates start in the preclinical phase, and only one so far has experience with licensure (Table \ref{tab:inex}), we take \(\omega = 0.92\);

\[
T_{i} = (1+I)\left(\omega T_i^{(n)} + (1-\omega)T_i^{(e)}\right)
\]

where \(I = 0.28\) is inflation from 2018 to 2025. Then the total weighted cost for phases 0 through 2 for \(N^{\text{(BPSV)}}\) candidates is

\begin{equation}
D_s^{\text{(BP-adRD)}} = \begin{cases}
 \left(N^{\text{(BPSV)}}-N^{\text{(BPSV-1)}}\right)\sum_{i=0}^2 \hat{P}_i^{(0)}T_{i} + N^{\text{(BPSV-1)}}\sum_{i=1}^2 \hat{P}_i^{(1)}T_{i} \; & \; s=1 \\
0  \; & \; s\neq 1
\end{cases}
\label{eq:bpsvrd}\end{equation}

\begin{longtable}[]{@{}ll@{}}
\caption{\label{tab:inex} Manufacturers working on BPSV and whether or not they have licensure experience}\tabularnewline
\toprule\noalign{}
Developer & Licensure Experience \\
\midrule\noalign{}
\endfirsthead
\toprule\noalign{}
Developer & Licensure Experience \\
\midrule\noalign{}
\endhead
\bottomrule\noalign{}
\endlastfoot
CalTech & No \\
SK Bio & Yes \\
Codiak & No \\
Panacea & No \\
NEC Onco & No \\
Intravacc & No \\
VIDO & No \\
IVI & No \\
\end{longtable}

\pandocbounded{\includegraphics[keepaspectratio,alt={\label{fig:posbpsv}Risk-adjusted R\&D cost for 8 BPSV candidates}]{README_files/figure-latex/posbpsv-1.pdf}} Min. 1st Qu. Median Mean 3rd Qu. Max.
80.68 223.61 297.93 319.18 378.17 1234.67

Target: 146 (103 135 177)

\subsection{BPSV investigational reserve}\label{bpsv-investigational-reserve}

\textbf{The annual cost annual is correct, at around 162 thousand, but the total cost is slightly too high}

Denoting the duration of each Phase \(i\) as \(Y_i^{(B)}\), the time taken to complete development of the BPSV up to the end of phase II, from which point it is manufactured to be held in an investigational reserve, is:

\[
Y^{(B)} = Y_0^{(B)} + Y_1^{(B)} + Y_2^{(B)}.
\]

The upfront cost of securing the investigational reserve is

\begin{equation}
D_{s,y}^{\text{(BP-man)}} = \begin{cases}
 A_4A_5
\; & \; s=1 \;\&\;y=Y^{(B)}+1\\
0  \; & \; s\neq 1\;\|\;y\neq Y^{(B)}+1
\end{cases}
\label{eq:bpsvadman}\end{equation}

where \(A_4 =100,000\) is the size of the reserve and \(A_5 =0.115\) is the cost per dose in USD.

The cost of goods supplied is \(G = 4.68\). Then the cost of drug substance, accounting for the fill/finish cost \(M_f = 0.14\) and the profit margin \(M_p = 0.2\), is \(G(1-M_f)(1+M_p) = 4.83\) USD per dose. The reserve is replenished every \(Y_{rep} = 3\) years. Then the annual cost to maintain the reserve of \(A_4 =100,000\) doses is

\begin{equation}
D_{s,y}^{\text{(BP-inv)}} = \begin{cases}
 \frac{A_4}{Y_{rep}}G  (1-M_f)(1+M_p) + A_1
\; & \; s=1 \;\&\;y>Y^{(B)}\\
0  \; & \; s\neq 1\;\|\;y\leq Y^{(B)}
\end{cases}
\label{eq:bpsvinv}\end{equation}

where \(A_1 = 0.01\) USD is the annual reservation cost per dose.

\pandocbounded{\includegraphics[keepaspectratio,alt={\label{fig:bpsvinv}BPSV investigational reserve costs accumulated from the completion of Phase II to year 15 with uniformly distributed discount rate.}]{README_files/figure-latex/bpsvinv-1.pdf}} Min. 1st Qu. Median Mean 3rd Qu. Max.
777.2 969.4 1084.5 1088.5 1196.7 1463.7

Target: 1 (0.9 1 1.1)

\subsection{SSV capacity reservation}\label{ssv-capacity-reservation}

\textbf{This matches the spreadsheet results.}

The cost per dose reservation per year is \(A_2 = 0.53\) USD. Reservation sizes, in billions, depend on scenarios, including the \(A_3 = 0.5\) billion doses reserved for HIC, as follows:

\[
M_{R,s} = \begin{cases}A_3 & s\in\{0, 1, 4, 7, 10\} \\ 
A_3+0.7 & s\in\{2, 5, 8\} \\ 
A_3+2 & s\in\{3, 6, 9\} \end{cases}
\]

Then the total cost per year is

\begin{equation}
D_s^{\text{(S-cap)}} =  M_{R,s} A_2
\label{eq:ssvcap}\end{equation}

The annual costs in billion USD are 0.27, 0.64, and 1.33, respectively.

\pandocbounded{\includegraphics[keepaspectratio,alt={\label{fig:capres}Capacity reservation costs accumulated over 15 years with uniformly distributed discount rate.}]{README_files/figure-latex/capres-1.pdf}} 0
Min. 1st Qu. Median Mean 3rd Qu. Max.
2737 2895 3065 3083 3272 3483

0.7
Min. 1st Qu. Median Mean 3rd Qu. Max.
6570 6948 7356 7399 7853 8360

2
Min. 1st Qu. Median Mean 3rd Qu. Max.
13687 14475 15326 15415 16361 17416

Targets:
3,086 (2,897 3,074 3,269)

7,407 (6,954 7,378 7,845)

15,431 (14,487 15,370 16,344)

\subsection{Enabling activities}\label{enabling-activities}

\textbf{This matches the spreadsheet results}

Denote the target ``days to SSV'' by \(\zeta\), so that \(\zeta\in\lbrace 365, 200, 100 \rbrace\), which are the three variations we consider enabling activities might achieve. We assume that \(Y^{(200)}=5\) years are required to achieve 200 days to SSV, and \(Y^{(100)}=15\) years are required to achieve 200 days to SSV. Then annual costs, \(E=700\) million, accumulate depending on the year and the mission:

\begin{equation}
D_{s,y}^{\text{(en)}} = \begin{cases}E & \zeta(s)=200 \;\&\; y\leq Y^{(200)} \; |\; \zeta(s)=100\; \& \;y\leq Y^{(100)} \\ 
0 & \zeta(s)=365 \;|\; y > Y^{(100)} \;|\; \zeta(s)=200 \;\&\; y \;>\; Y^{(200)}  \end{cases}
\label{eq:enable}\end{equation}

For our scenarios, we have

\[
\zeta(s) = \begin{cases} 365 & s\in\{0, 1, 2, 3, 10\} \\ 
200 & s\in\{4, 5, 6\} \\ 
100 & s\in\{7, 8, 9\} \end{cases}
\]

\pandocbounded{\includegraphics[keepaspectratio,alt={\label{fig:en}Enabling costs accumulated over 15 years with uniformly distributed discount rate.}]{README_files/figure-latex/en-1.pdf}} 365
Min. 1st Qu. Median Mean 3rd Qu. Max.
0 0 0 0 0 0

200
Min. 1st Qu. Median Mean 3rd Qu. Max.
3.13 3.18 3.24 3.24 3.30 3.37

100
Min. 1st Qu. Median Mean 3rd Qu. Max.
7.21 7.62 8.07 8.12 8.62 9.17

Targets:

3,242 (3,182 3,241 3,302)

8,126 (7,629 8,094 8,607)

\section{Response cost equation}\label{response-cost-equation}

We write total response costs as

\[
D_{s,y}^{\text{(res)}} = \frac{1}{(1+r)^y}\left(D_{s,y}^{\text{(BP-resRD)}} + D_{s,y}^{\text{(S-RD)}} + D_{s,y}^{\text{(BP-proc)}} + D_{s,y}^{\text{(S-proc)}} + D_{s,y}^{\text{(BP-del)}} + D_{s,y}^{\text{(S-del)}}\right)
\]

where

\begin{itemize}
\tightlist
\item
  \(D_{s,y}^{\text{(BP-resRD)}}\) is the R\&D cost of BPSV after an outbreak; see Equation \eqref{eq:bpsvresrd}
\item
  \(D_{s,y}^{\text{(S-RD)}}\) is the R\&D cost for SSV; see Equation \eqref{eq:ssvrd}
\item
  \(D_{s,y}^{\text{(BP-proc)}}\) is the cost of procuring BPSV; see Equation \eqref{eq:bpsvproc}
\item
  \(D_{s,y}^{\text{(S-proc)}}\) is the cost of procuring SSV; see Equation \eqref{eq:ssvproc}
\item
  \(D_{s,y}^{\text{(BP-del)}}\) is the cost of delivering BPSV; see Equation \eqref{eq:bspvdel}
\item
  \(D_{s,y}^{\text{(S-del)}}\) is the cost of delivering SSV; see Equation \eqref{eq:ssvdel}
\end{itemize}

\subsection{Risk-adjusted R\&D cost per candidate calculation}\label{risk-adjusted-rd-cost-per-candidate-calculation}

\subsubsection{SSV}\label{ssv}

\textbf{These don't match the spreadsheet results. Values too low.}

Trial costs are adjusted for the duration of the trial, which depend on the R\&D investment, denoted \(\zeta\in\lbrace 365, 200, 100\rbrace\):

\[
T_{\zeta,i}^{(e)} = (1+I)\frac{W_{i;\zeta}^{(S)}}{52Y_{i}^{(B)}}T_i^{(e)}
\]

where \(I = 0.28\) is inflation from 2018 to 2025, \(T_i^{(e)}\) is the cost per Phase \(i\) of experienced developers, \(Y_{i}^{(B)}\) is the expected phase duration in years, and \(W_{i;\zeta}^{(S)}\) is its expected duration in weeks given enabling investments made prior to the outbreak.

The probability of success of each phase comes from COVID-19 data.

\[
P_i^{\text{(SSV)}} \sim \text{Beta}\left(\sum_{j=i+1}^4X_j+1, X_i + 1 \right)
\]

for \(i \in \lbrace 0,1,2,3 \rbrace\) where \(X_i\) is the number of candidates that failed in phase \(i\) and \(X_4\) the number that succeeded to licensure.

The probabilities of phase occurrence are:

\[
\hat{P}_i^{\text{(SSV)}} = \begin{cases}1 & i=0 \\ 
\prod_{j=0}^{i-1}P_j^{\text{(SSV)}} & i\in\lbrace 1,2,3,4\rbrace 
\end{cases}
\]

Then the total cost is

\begin{equation}
D_s^{\text{(S-RD)}} = N^{\text{(SSV)}}\left(\sum_{i=0}^3 \hat{P}_i^{\text{(SSV)}} \cdot T_{\zeta(s),i}^{(e)} +  \hat{P}_4^{\text{(SSV)}} T_4\right)
\label{eq:ssvrd}\end{equation}

where \(T_4\) is the cost of licensure. We multiply by the number of candidates, \(N^{\text{(SSV)}}\), to get the total cost from the weighted average per candidate, where

\[
N^{\text{(SSV)}} = n^{\text{(SSV)}} + F^{-1}_{NegBin}\left(Q^{\text{(SSV)}}; n^{\text{(SSV)}},  \hat{P}_4^{\text{(SSV)}} \right)
\]

is chosen to secure at least \(n^{\text{(SSV)}} = 5\) successful candidates with probability \(Q^{\text{(SSV)}} = 90\)\%. Here, \(F^{-1}_{NegBin}\left(q; n,  p \right)\) is the cumulative density of a negative binomial distribution with parameters \(n\) and \(p\) evaluated at quantile \(q\).

\subsection{\texorpdfstring{\protect\pandocbounded{\includegraphics[keepaspectratio,alt={\label{fig:posssv}Risk-adjusted R\&D cost for 18 SSV candidates}]{README_files/figure-latex/posssv-1.pdf}}}{\label{fig:posssv}Risk-adjusted R\&D cost for 18 SSV candidates}}\label{figposssvrisk-adjusted-rd-cost-for-18-ssv-candidates}

\begin{longtable}[]{@{}ccccccc@{}}
\toprule\noalign{}
DM & Min. & 1st Qu. & Median & Mean & 3rd Qu. & Max. \\
\midrule\noalign{}
\endhead
\bottomrule\noalign{}
\endlastfoot
365 & 118 & 172 & 203 & 208 & 237 & 386 \\
\end{longtable}

200 60 85 100 103 117 180

\subsection{100 35 50 59 60 69 113}\label{section}

Targets:

284 (105 170 283)

195 (61 97 164)

118 (35 61 108)

\subsubsection{BPSV}\label{bpsv}

\textbf{This is a little higher than the spreadsheet results}

One BPSV candidate that has passed through phases 0 to 2 prior to the outbreak goes through Phase III during the response. The duration is \(W_3^{(B)}=18\) weeks. Thus we write the BPSV R\&D response cost

\begin{equation}
D_s^{\text{(BP-resRD)}} = \begin{cases}\left( (1+I)\frac{W_3^{(B)}}{52Y_3^{(B)}}T_3^{(e)} + P_3 T_4\right) \; & \; s=1 \\
0  \; & \; s\neq 1
\end{cases}
\label{eq:bpsvresrd}\end{equation}

\pandocbounded{\includegraphics[keepaspectratio,alt={\label{fig:bpsvresrd}Reactive R\&D cost for BPSV}]{README_files/figure-latex/bpsvresrd-1.pdf}} Min. 1st Qu. Median Mean 3rd Qu. Max.
0.5 2.0 3.7 8.4 7.1 936.9

Target: 14 (3 5 10)

\subsection{Procurement cost calculation}\label{procurement-cost-calculation}

The cost per dose comes from the cost of goods supplied (\(G = 4.68\)) adjusted for profits (\(M_p = 0.2\)) and the transportation cost (\(M_t = 0.12\)).

\(S_R = G(1+M_p)(1+M_t)\) evaluates to 6.29.

This cost is used both for SSV doses manufactured using reserved capacity, and all newly manufactured BPSV doses.

\subsubsection{SSV}\label{ssv-1}

\textbf{These values are close, but not identical, to the spreadsheet results if I adjust for the total demand}

We write billion doses procured from channel \(x\in\lbrace R,E,B \rbrace\) in year \(y\) and scenario \(s\) as \(A_{x,s,y}\) (see Equation \eqref{eq:supply}). Then the total cost, in billion USD, is:

\begin{equation}
D_{s,y}^{\text{(S-proc)}} = A_{R,s,y} S_R  + \sum_{x\in\lbrace E,B \rbrace}  A_{x,s,y} S_U
\label{eq:ssvproc}\end{equation}

Here, \(S_R = 6.29\) is the cost per reserved dose and \(S_U = 18.94\) the cost per unreserved dose in USD.

\subsection{\texorpdfstring{\protect\pandocbounded{\includegraphics[keepaspectratio,alt={\label{fig:costperyear}SSV procurement cost}]{README_files/figure-latex/costperyear-1.pdf}}}{\label{fig:costperyear}SSV procurement cost}}\label{figcostperyearssv-procurement-cost}

\begin{longtable}[]{@{}ccccccc@{}}
\toprule\noalign{}
Scenario & Min. & 1st Qu. & Median & Mean & 3rd Qu. & Max. \\
\midrule\noalign{}
\endhead
\bottomrule\noalign{}
\endlastfoot
BAU & 134252 & 159130 & 188416 & 193659 & 227378 & 270758 \\
\end{longtable}

S01 137394 162408 191773 196956 230728 273971

S02 126163 149484 176927 181830 213423 254038

S03 99526 117747 139155 142949 167579 199157

S04 140126 165219 194599 199714 233466 276490

S05 125310 147707 173924 178481 208595 246964

S06 104828 123480 145299 149078 174133 206019

S07 140537 165492 194675 199724 233233 275861

S08 124746 146834 172653 177110 206751 244431

S09 102454 120478 141527 145142 169297 199951

\subsection{S10 130381 154522 182937 188021 220736 262815}\label{s10-130381-154522-182937-188021-220736-262815}

Table: Costs summed and discounted from year 16 to year 20, million USD

Targets:

184,127 ( 151,271 180,171 214,966 )
187,255 ( 154,376 183,358 218,147 )
167,519 ( 137,713 163,938 195,495 )
135,910 ( 111,925 133,050 158,444 )
189,820 ( 157,000 185,976 220,684 )
169,549 ( 140,293 166,133 197,067 )
141,440 ( 117,134 138,613 164,309 )
189,878 ( 157,295 186,091 220,526 )
168,378 ( 139,564 165,035 195,494 )
137,984 ( 114,513 135,278 160,078 )
178,766 ( 146,883 174,927 208,686 )

\subsubsection{BPSV}\label{bpsv-1}

\textbf{This is pretty close}

The cost of BPSV doses is the sum of new doses supplied, \(A_{BPSV,s}\), at the reserved-capacity cost per dose, and the doses held in the investigational reserve, for which fill/finish, transport and profit margin costs are due.

\begin{equation}
D_s^{\text{(BP-proc)}} = \begin{cases}
A_{BPSV,s}\cdot S_R +  A_4(M_f+M_t)(1+M_p)G\; & \; s=1 \\
0  \; & \; s\neq 1
\end{cases}
\label{eq:bpsvproc}\end{equation}

For a world population aged 65 and over of 0.86 billion, an uptake of 80\% (accounting for wastage of 31\%), and a cost per dose of \(S_R = 6.29\) USD (the same as for SSV via reserved capacity), the procurement cost for BPSV is 6.68 billion USD.

In our model, 1.0625 billion doses are manufactured, as manufacturing stops once one billion doses have been made.

Min. 1st Qu. Median Mean 3rd Qu. Max.
2790 3208 3685 3758 4300 4962

Target: 3,628 (3,062 3,568 4,165)

\subsection{Delivery Cost Equation}\label{delivery-cost-equation}

\subsubsection{SSV}\label{ssv-2}

\textbf{These values are ballpark correct but too concentrated}

For populations aged 15 and above \(N_i^{(15)}\), we write

\[
L_i = 2\cdot \lambda\cdot N_i^{(15)}
\]

as the total demand for first-schedule doses for income group \(i\in\lbrace\text{LIC, LMIC, UMIC, HIC}\rbrace\), representing two doses each for \(\lambda = 80\)\% of the population.

We write the delivery cost for \(h_{s,i,w}\) doses given in week \(w\) and country type \(i\) as follows. There are three cost tiers, the first of which, \(V_{i; 0}\), is applied to the first 10\% of \(L_i\), the second (\(V_{i; 10}\)) to the subsequent 20\%, and the third (\(V_{i; 30}\)) to all doses thereafter. The same costing schedule applies both to the first-schedule plus booster SSV doses and the BPSV rollout.

\[
H_{s,i,w} = 
\begin{cases}
V_{i; 0}h_{s,i,w}  & \sum_{j=1}^{w-1}h_{s,i,w}\leq \frac{1}{10}L_i \\
V_{i; 10}h_{s,i,w}  & \frac{1}{10}L_i < \sum_{j=1}^{w-1}h_{s,i,w}\leq \frac{3}{10}L_i  \\
V_{i; 30}h_{s,i,w}  & \frac{3}{10}L_i < \sum_{j=1}^{w-1}h_{s,i,w}
\end{cases}
\]

Then the delivery cost in year \(y\) and scenario \(s\) is

\begin{equation}
D_{s,y}^{\text{(S-del)}} = 
\sum_{w\in y} \sum_i H_{s,i,w}
\label{eq:ssvdel}\end{equation}

\pandocbounded{\includegraphics[keepaspectratio,alt={\label{fig:deliverycost}SSV delivery cost}]{README_files/figure-latex/deliverycost-1.pdf}}\\
BAU Min. : 64160 1st Qu.: 94621 Median :110310 Mean :114254\\
S01 Min. : 64505 1st Qu.: 94996 Median :110672 Mean :114586\\
S02 Min. : 64240 1st Qu.: 94710 Median :110394 Mean :114330\\
S03 Min. : 64342 1st Qu.: 94820 Median :110498 Mean :114424\\
S04 Min. : 64809 1st Qu.: 95402 Median :111018 Mean :114916\\
S05 Min. : 65090 1st Qu.: 95726 Median :111324 Mean :115187\\
S06 Min. : 65262 1st Qu.: 95925 Median :111522 Mean :115369\\
S07 Min. : 66504 1st Qu.: 97669 Median :113239 Mean :117044\\
S08 Min. : 65980 1st Qu.: 96758 Median :112249 Mean :116167\\
S09 Min. : 66202 1st Qu.: 96989 Median :112466 Mean :116407\\
S10 Min. : 63488 1st Qu.: 93615 Median :109264 Mean :113248

\begin{verbatim}
 BAU 3rd Qu.:130779   Max.   :196492  
 S01 3rd Qu.:131087   Max.   :196757  
 S02 3rd Qu.:130848   Max.   :196550  
 S03 3rd Qu.:130938   Max.   :196624  
 S04 3rd Qu.:131399   Max.   :197057  
 S05 3rd Qu.:131681   Max.   :197276  
 S06 3rd Qu.:131891   Max.   :197435  
 S07 3rd Qu.:133474   Max.   :199217  
 S08 3rd Qu.:132690   Max.   :198177  
 S09 3rd Qu.:132874   Max.   :198397  
 S10 3rd Qu.:129996   Max.   :195357  
\end{verbatim}

Targets:

114,526 ( 90,654 110,005 134,444 )
114,771 ( 91,321 111,130 134,341 )
114,769 ( 91,620 110,604 133,752 )
114,811 ( 91,647 110,815 133,856 )
114,527 ( 91,170 110,664 133,720 )
114,615 ( 91,074 110,653 133,836 )
115,095 ( 91,858 111,205 134,355 )
115,634 ( 92,514 111,639 134,375 )
116,385 ( 93,116 112,183 135,664 )
117,196 ( 93,427 113,114 136,861 )
116,913 ( 93,536 112,957 136,414 )
118,141 ( 94,682 114,649 137,100 )
113,540 ( 89,745 109,012 132,595 )

\subsubsection{BPSV}\label{bpsv-2}

\textbf{These values match the spreadsheet results. (NB: more doses are purchased and delivered than there are eligible people in the population)}

For the BPSV, which goes only to people aged 65 or older, with populations \(N_i^{(65)}\), coverage is reached earlier in the process, so the cost is weighted more heavily towards start up and ramp up:

\begin{equation}
D_s^{\text{(BP-del)}} = 
\begin{cases}
\sum_{i}D_{\text{BPSV},i}
\; & \; s=1 \\
0  \; & \; s\neq 1
\end{cases}
\label{eq:bspvdel}\end{equation}

\[
D_{\text{BPSV},i} = 
\begin{cases}
N_i^{(65)}V_{i; 0}  & N_i^{(65)}\leq \frac{1}{10}N_i^{(15)} \\
\frac{N_i^{(15)}}{10} V_{i; 0} + \left(N_i^{(65)}-\frac{N_i^{(15)}}{10} \right)V_{i; 11}  & \frac{1}{10}N_i^{(15)} < N_i^{(65)}\leq \frac{3}{10}N_i^{(15)} \\
\frac{N_i^{(15)}}{10} V_{i; 0} + \frac{2}{10}N_i^{(15)} V_{i; 11} + \left(N_i^{(65)}-\frac{3}{10}N_i^{(15)} \right)V_{i; 31} & N_i^{(65)}> \frac{3}{10} N_i^{(15)}
\end{cases}
\]

The logic of this is as follows:

\begin{itemize}
\tightlist
\item
  The increments in cost correspond to numbers of eligible people in the whole population, namely those aged 15 and above.
\item
  If the number of people eligible for the BPSV is less than 10\% of the population aged 15 and over, then all doses cost the ``start up'' amount.
\item
  If the number of people eligible for the BPSV is more than 10\% and less than 30\% of the 15+ population, then cost of the first doses, a number equal to 10\% of the 15+ population, is the ``start up'' amount. All remaining doses cost the ``ramp up'' amount.
\item
  If the number of people eligible for the BPSV is more than 30\% of the 15+ population, then the cost of the first doses, a number equal to 10\% of the 15+ population, is the ``start up'' amount. The cost of the second tranche of doses, a number equal to 20\% of the 15+ population, is the ``ramp up'' amount. All remaining doses cost the ``getting to scale'' amount.
\end{itemize}

\pandocbounded{\includegraphics[keepaspectratio,alt={\label{fig:bpsvdeliverycost}BPSV delivery cost}]{README_files/figure-latex/bpsvdeliverycost-1.pdf}} Min. 1st Qu. Median Mean 3rd Qu. Max.
6379 9907 11394 11614 13150 19017

Target: 11,206 (9,037 10,865 13,054)

\begin{longtable}[]{@{}
  >{\raggedright\arraybackslash}p{(\linewidth - 8\tabcolsep) * \real{0.2000}}
  >{\raggedright\arraybackslash}p{(\linewidth - 8\tabcolsep) * \real{0.2000}}
  >{\raggedright\arraybackslash}p{(\linewidth - 8\tabcolsep) * \real{0.2000}}
  >{\raggedright\arraybackslash}p{(\linewidth - 8\tabcolsep) * \real{0.2000}}
  >{\raggedright\arraybackslash}p{(\linewidth - 8\tabcolsep) * \real{0.2000}}@{}}
\caption{\label{tab:delcosts} Literature review of global and country-specific delivery costs}\tabularnewline
\toprule\noalign{}
\begin{minipage}[b]{\linewidth}\raggedright
Country
\end{minipage} & \begin{minipage}[b]{\linewidth}\raggedright
Country status
\end{minipage} & \begin{minipage}[b]{\linewidth}\raggedright
Study type
\end{minipage} & \begin{minipage}[b]{\linewidth}\raggedright
Financial Cost per dose (USD)
\end{minipage} & \begin{minipage}[b]{\linewidth}\raggedright
Source
\end{minipage} \\
\midrule\noalign{}
\endfirsthead
\toprule\noalign{}
\begin{minipage}[b]{\linewidth}\raggedright
Country
\end{minipage} & \begin{minipage}[b]{\linewidth}\raggedright
Country status
\end{minipage} & \begin{minipage}[b]{\linewidth}\raggedright
Study type
\end{minipage} & \begin{minipage}[b]{\linewidth}\raggedright
Financial Cost per dose (USD)
\end{minipage} & \begin{minipage}[b]{\linewidth}\raggedright
Source
\end{minipage} \\
\midrule\noalign{}
\endhead
\bottomrule\noalign{}
\endlastfoot
WHO, Gavi, and UNICEF AMC Estimate & AMC & Top down & 1.66 & \citet{Griffiths2021} \\
UNICEF Global Estimate & All & Model & 0.73 & \citet{Oyatoye2023} \\
DRC & LIC & Bottom up & 1.91 & \citet{Moi2024} \\
Malawi & LIC & Bottom up & 4.55 & \citet{Ruisch2025} \\
Mozambique & LIC & Bottom up & 0.5 & \citet{Namalela2025} \\
Uganda & LIC & Bottom up & 0.79 & \citet{Tumusiime2024} \\
Bangladesh & LMIC & Bottom up & 0.29 & \citet{Yesmin2024} \\
Cote d'Ivoire & LMIC & Bottom up & 0.67 & \citet{Vaughan2023} \\
Nigeria & LMIC & Bottom up & 0.84 & \citet{Noh2024} \\
Philippines & LMIC & Bottom up & 2.16 & \citet{Banks2023} \\
Vietnam & LMIC & Bottom up & 1.73 & \citet{Nguyen2024} \\
Ghana & LMIC & CVIC tool & 2.2--2.3 & \citet{Nonvignon2022} \\
Lao PDR & LMIC & CVIC tool & 0.79--0.81 & \citet{Yeung2023} \\
Kenya & LMIC & Top down & 3.29--4.28 & \citet{Orangi2022} \\
Botswana & UMIC & Mixed & 19 & \citet{Vaughan2025} \\
South Africa & UMIC & Top down & 3.84 & \citet{Edoka2024} \\
\end{longtable}

\begin{longtable}[]{@{}
  >{\centering\arraybackslash}p{(\linewidth - 6\tabcolsep) * \real{0.1829}}
  >{\centering\arraybackslash}p{(\linewidth - 6\tabcolsep) * \real{0.1951}}
  >{\centering\arraybackslash}p{(\linewidth - 6\tabcolsep) * \real{0.2805}}
  >{\centering\arraybackslash}p{(\linewidth - 6\tabcolsep) * \real{0.3415}}@{}}
\caption{Cost differences: investments vs.~BAU, for different types of investment.}\tabularnewline
\toprule\noalign{}
\begin{minipage}[b]{\linewidth}\centering
timing
\end{minipage} & \begin{minipage}[b]{\linewidth}\centering
category
\end{minipage} & \begin{minipage}[b]{\linewidth}\centering
type
\end{minipage} & \begin{minipage}[b]{\linewidth}\centering
Cost vs.~BAU
\end{minipage} \\
\midrule\noalign{}
\endfirsthead
\toprule\noalign{}
\begin{minipage}[b]{\linewidth}\centering
timing
\end{minipage} & \begin{minipage}[b]{\linewidth}\centering
category
\end{minipage} & \begin{minipage}[b]{\linewidth}\centering
type
\end{minipage} & \begin{minipage}[b]{\linewidth}\centering
Cost vs.~BAU
\end{minipage} \\
\midrule\noalign{}
\endhead
\bottomrule\noalign{}
\endlastfoot
Upfront & Manufacturing & BPSV & \(0.000012\) \\
Upfront & R\&D & BPSV & \(0.31\) (\(0.24,0.4\)) \\
Upfront & R\&D & 200 days to SSV & \(3.5\) \\
Upfront & R\&D & 100 days to SSV & \(10\) \\
Per year & Manufacturing & BPSV & \(0.16\) \\
Per year & Manufacturing & 0.7 billion capacity & \(0.37\) \\
Per year & Manufacturing & 2 billion capacity & \(1.1\) \\
Per pandemic & Delivery & BPSV & \(20\) (\(19,22\)) \\
Per pandemic & Delivery & 0.7 billion capacity & \(0.14\) (\(0.11,0.17\)) \\
Per pandemic & Delivery & 2 billion capacity & \(0.32\) (\(0.24,0.38\)) \\
Per pandemic & Delivery & Equality + Delivery & \(-1.8\) (\(-2.3,-1.4\)) \\
Per pandemic & Manufacturing & BPSV & \(6.7\) \\
Per pandemic & Manufacturing & 0.7 billion capacity & \(-23\) \\
Per pandemic & Manufacturing & 2 billion capacity & \(-99\) \\
Per pandemic & R\&D & BPSV & \(0.0064\) (\(0.0037,0.013\)) \\
Per pandemic & R\&D & 200 days to SSV & \(-0.18\) (\(-0.2,-0.17\)) \\
Per pandemic & R\&D & 100 days to SSV & \(-0.26\) (\(-0.28,-0.24\)) \\
\end{longtable}

\section{SSV delivery}\label{ssv-delivery}

\begin{longtable}[]{@{}
  >{\raggedright\arraybackslash}p{(\linewidth - 6\tabcolsep) * \real{0.2500}}
  >{\raggedright\arraybackslash}p{(\linewidth - 6\tabcolsep) * \real{0.2500}}
  >{\raggedright\arraybackslash}p{(\linewidth - 6\tabcolsep) * \real{0.2500}}
  >{\raggedright\arraybackslash}p{(\linewidth - 6\tabcolsep) * \real{0.2500}}@{}}
\caption{Manufacturing response timeline assumptions}\tabularnewline
\toprule\noalign{}
\begin{minipage}[b]{\linewidth}\raggedright
Category
\end{minipage} & \begin{minipage}[b]{\linewidth}\raggedright
Reserved capacity
\end{minipage} & \begin{minipage}[b]{\linewidth}\raggedright
Private response (existing capacity)
\end{minipage} & \begin{minipage}[b]{\linewidth}\raggedright
Private response (built capacity)
\end{minipage} \\
\midrule\noalign{}
\endfirsthead
\toprule\noalign{}
\begin{minipage}[b]{\linewidth}\raggedright
Category
\end{minipage} & \begin{minipage}[b]{\linewidth}\raggedright
Reserved capacity
\end{minipage} & \begin{minipage}[b]{\linewidth}\raggedright
Private response (existing capacity)
\end{minipage} & \begin{minipage}[b]{\linewidth}\raggedright
Private response (built capacity)
\end{minipage} \\
\midrule\noalign{}
\endhead
\bottomrule\noalign{}
\endlastfoot
Annual manufacturing volume & By scenario (0.5--2.5B) & 9B minus reserved volume & 6B \\
Facility transition start & 7 weeks before vaccine approval & 7 weeks before vaccine approval & 7 weeks before vaccine approval \\
Weeks to initial manufacturing & 12 & 12 (BPSV) or 30 (no BPSV) & 48 \\
Scale-up weeks to full capacity & 10 & 16 & 16 \\
\end{longtable}

\begin{longtable}[]{@{}
  >{\raggedright\arraybackslash}p{(\linewidth - 6\tabcolsep) * \real{0.2500}}
  >{\raggedright\arraybackslash}p{(\linewidth - 6\tabcolsep) * \real{0.2500}}
  >{\raggedright\arraybackslash}p{(\linewidth - 6\tabcolsep) * \real{0.2500}}
  >{\raggedright\arraybackslash}p{(\linewidth - 6\tabcolsep) * \real{0.2500}}@{}}
\caption{Vaccine Production Timeline when there is no BPSV. When BPSV is also modelled, Existing Private Capacity scales from 0 to 100 in weeks 12--21.}\tabularnewline
\toprule\noalign{}
\begin{minipage}[b]{\linewidth}\raggedright
Weeks from transition start
\end{minipage} & \begin{minipage}[b]{\linewidth}\raggedright
Reserved Capacity (\%)
\end{minipage} & \begin{minipage}[b]{\linewidth}\raggedright
Existing Private Capacity (\%)
\end{minipage} & \begin{minipage}[b]{\linewidth}\raggedright
Response Private Capacity (\%)
\end{minipage} \\
\midrule\noalign{}
\endfirsthead
\toprule\noalign{}
\begin{minipage}[b]{\linewidth}\raggedright
Weeks from transition start
\end{minipage} & \begin{minipage}[b]{\linewidth}\raggedright
Reserved Capacity (\%)
\end{minipage} & \begin{minipage}[b]{\linewidth}\raggedright
Existing Private Capacity (\%)
\end{minipage} & \begin{minipage}[b]{\linewidth}\raggedright
Response Private Capacity (\%)
\end{minipage} \\
\midrule\noalign{}
\endhead
\bottomrule\noalign{}
\endlastfoot
0--11 & & & \\
12--21 & Scales from 0 to 100 & & \\
22--29 & 100 & & \\
30--45 & 100 & Scales from 0 to 100 & \\
46--47 & 100 & 100 & \\
48--63 & 100 & 100 & Scales from 0 to 100 \\
64+ & 100 & 100 & 100 \\
\end{longtable}

\subsection{Timing}\label{timing}

Facility transition occurs \(I_0=7\) weeks before vaccine approval, which in turn depends on R\&D investments. We have three levels in our scenarios, corresponding to SSVs available in 100 days, 200 days, and 365 days. The total weeks taken for vaccine approval can be written as follows:

\[
W_{\zeta}^{(S)} = \sum_{i=0}^3 W_{i;\zeta}^{(S)}
\]

for \(\zeta\in\lbrace 365, 200, 100\rbrace\). These work out as 52, 29, and 14 weeks, respectively. Thus ``week 0'' for manufacturing occurs 45, 22, and 7 weeks, respectively, after the new pathogen has been sequenced. We denote this variable \(w_s^{(0)}\):

\[
w_s^{(0)} = W_{\zeta(s)}^{(S)} - 7 = \begin{cases}45 & s\in\lbrace 0, 1, 2, 3, 10\rbrace \\ 
22 & s\in\lbrace4, 5, 6\rbrace \\ 
7 & s\in\lbrace7, 8, 9\rbrace \end{cases}
\]

\subsection{Production}\label{production}

The total global manufacturing volume is \(M_G=15\) billion doses. The amount that is reserved, in billion doses, including the HIC-specific reservation of \(A_3=0.5\) billion doses, depends on the scenarios as follows:

\[
M_{R,s} = \begin{cases}A_3 & s\in\{0, 1, 4, 7, 10\} \\ 
A_3 + 0.7 & s\in\{2, 5, 8\} \\ 
A_3 + 2 & s\in\{3, 6, 9\} \end{cases}
\]

where \(s=0\) denotes the BAU scenario. By definition, \(M_{E,s} = M_C - M_{R,s}\) (existing unreserved capacity equals the total currently existing minus reserved capacity), and \(M_B=M_G-M_C\) (newly built manufacturing equals the global total minus the existing capacity).

Then the number of doses, in billions, that are made from capacity \(x\in \lbrace R, E, B\rbrace\) in week \(w\) of scenario \(s\) is:

\[
Z_{x,s,w} = \begin{cases}0 & w-w_s^{(0)} \leq I_x \\ 
\frac{1}{52}\frac{w-w_s^{(0)}-I_x}{C_x}M_{x,s} & w-w_s^{(0)}\in(I_x, I_x+C_x] \\ 
\frac{1}{52}M_{x,s}  & w-w_s^{(0)}> I_x+C_x
\end{cases}
\]

where \(I_R = 12\) is the number of weeks to initial manufacturing for reserved capacity, and \(C_R = 10\) is its number of weeks to scale up to full capacity; \(I_B = 48\) is the number of weeks to initial manufacturing for newly built (and, therefore, unreserved) capacity, and \(C_B = 16\) is its number of weeks to scale up to full capacity; and

\[
I_E = \begin{cases}
 I_{E,1} \; & \; s=1 \\
I_{E,0}  \; & \; s\neq 1
\end{cases}
\]

where \(I_{E,0} = 30\) and \(I_{E,1} = 12\) are the number of weeks to initial manufacturing for existing and unreserved capacity, and \(C_E = 16\) is its number of weeks to scale up to full capacity.

\begin{figure}
\centering
\pandocbounded{\includegraphics[keepaspectratio,alt={\label{fig:supply}Doses made available from manufacturing per scenario. Weeks are in reference to the sequencing of the pathogen.}]{README_files/figure-latex/supply-1.pdf}}
\caption{\label{fig:supply}Doses made available from manufacturing per scenario. Weeks are in reference to the sequencing of the pathogen.}
\end{figure}

\subsection{Allocation}\label{allocation}

Denote the weekly allocated doses at week \(w\) from capacity \(x\) to income level \(i\) \(k_{s,x,i,w}\), and the cumulative number \(K_{s,i,w}\), such that

\[
K_{s,i,w} = \sum_{x\in\lbrace R,E,B\rbrace}\sum_{j=0}^w k_{s,x,i,j}.
\]

\[
k_{s,R,i,w} = \begin{cases}
\left(\frac{A_3}{M_{R,s}} + \frac{M_{R,s}-A_3}{M_{R,s}}\frac{N_{HIC}}{N_T}\right)Z_{R,s,w}             & K_{s,\text{HIC},w} <  L_{\text{HIC}}\;\&\; i=\text{HIC} \\
\frac{M_{R,s}-A_3}{M_{R,s}}\frac{N_{i}}{N_T}Z_{R,s,w}                    & K_{s,\text{HIC},w} < L_{\text{HIC}}\;\&\; i\neq\text{HIC}  \\
0 & K_{s,\text{HIC},w} \geq L_{\text{HIC}} \;\&\; i=\text{HIC}\\
\frac{N_{i}}{N_{UMIC}+N_{LMIC}+N_{LIC}}Z_{R,s,w} & K_{s,\text{HIC},w} \geq L_{\text{HIC}} \;\&\;  K_{s,\text{UMIC},w} < L_{\text{UMIC}} \;\&\; i\neq\text{HIC}\\
0 & K_{s,\text{UMIC},w} \geq L_{\text{UMIC}} \;\&\; i=\text{UMIC}\\
\frac{N_{i}}{N_{LMIC}+N_{LIC}}Z_{R,s,w} & K_{s,\text{UMIC},w} \geq L_{\text{UMIC}} \;\&\;  K_{s,\text{LMIC},w} < L_{\text{LMIC}} \;\&\; i\notin  \lbrace\text{HIC},\text{UMIC}\rbrace\\
0 & K_{s,\text{LMIC},w} \geq L_{\text{LMIC}} \;\&\; i=\text{LMIC}\\
Z_{R,s,w}             & K_{s,\text{LMIC},w} \geq L_{\text{LMIC}} \;\&\; K_{s,\text{LIC},w} < L_{\text{LIC}} \;\&\; i=\text{LIC}\\
0 & K_{s,\text{LIC},w} \geq L_{\text{LIC}} 
\end{cases}
\]

where \(N_T = \sum_{i\in\lbrace\text{HIC,UMIC,LMIC,LIC}\rbrace}N_i\).

The logic of this reads as follows:

\begin{itemize}
\tightlist
\item
  \(A_3=0.5\) billion doses per year from reserved capacity go exclusively to HIC, which is expressed as a fraction of the total reservation, \(M_{R,s}\)
\item
  Any remaining reserved capacity doses are allocated according to population
\item
  Once HIC reach their total demand, doses from reserved capacity are split proportional to population between UMIC, LMIC and LIC, and so on
\end{itemize}

For \(x\in\lbrace E,B\rbrace\),

\[
k_{s,x,i,w} = \begin{cases}
Z_{x,s,w}            & K_{s,\text{HIC},w} < L_{\text{HIC}} \;\&\; i=\text{HIC} \\
0                     & K_{s,\text{HIC},w} < L_{\text{HIC}} \;\&\; i\neq\text{HIC} \\
Z_{x,s,w}            & K_{s,\text{HIC},w} \geq L_{\text{HIC}} \;\&\; K_{s,\text{UMIC},w} < L_{\text{UMIC}} \;\&\; i=\text{UMIC} \\
0                     & K_{s,\text{HIC},w} \geq L_{\text{HIC}} \;\&\; K_{s,\text{UMIC},w} < L_{\text{UMIC}} \;\&\; i\neq\text{UMIC} \\
Z_{x,s,w}            & K_{s,\text{UMIC},w} \geq L_{\text{UMIC}} \;\&\; K_{s,\text{LMIC},w} < L_{\text{LMIC}} \;\&\; i=\text{LMIC} \\
0                     & K_{s,\text{UMIC},w} \geq L_{\text{UMIC}} \;\&\; K_{s,\text{LMIC},w} < L_{\text{LMIC}} \;\&\; i\neq\text{LMIC} \\
Z_{x,s,w}            & K_{s,\text{LMIC},w} \geq L_{\text{LMIC}} \;\&\; i=\text{LIC} \\
0                     & K_{s,\text{LMIC},w} \geq L_{\text{LMIC}} \;\&\; i\neq\text{LIC} 
\end{cases}
\]

The logic of this reads as follows:

\begin{itemize}
\tightlist
\item
  Until HIC demand is reached, all doses from unreserved capacity go to HIC. None go to UMIC, LMIC and LIC.
\item
  Once HIC demand has been met and until UMIC demand is reached, all doses from unreserved capacity go to UMIC. None go to HIC, LMIC and LIC.
\item
  Once HIC and UMIC demand have been met and until LMIC demand is reached, all doses from unreserved capacity go to LMIC. None go to HIC, UMIC and LIC.
\item
  Once HIC, UMIC and LMIC demand have been met, all remaining doses from unreserved capacity go to LIC. None go to LMIC, UMIC and HIC.
\end{itemize}

Total supply of first-schedule doses in each year period is

\[
A_{x,s,y}^{(1)} = \sum_i\sum_{w\in y}k_{s,x,i,w}.
\]

We assume, for every second dose of SSV, a booster will be given one year later for \(N^{\text{(boost)}} = 2\) years.

Thus

\[
A_{s,y}^{(2)} = \begin{cases}
\frac{1}{2}\sum_{x}A_{x,s,y-1}^{(1)}            & y=2 \\
\frac{1}{2}\sum_{x}\left(A_{x,s,y-1}^{(1)} + A_{x,s,y-2}^{(1)}\right)            & y>2 
\end{cases}
\]

and

\[
A_{x,s,y}^{(2)} = \min(A_{s,y}^{(2)}, M_R - A_{R,s,y}^{(1)})
\]

for \(x=R\) and

\[
A_{x,s,y}^{(2)} = \max(A_{s,y}^{(2)} - A_{R,s,y}^{(2)}, 0)
\]

for \(x\in\lbrace E, B \rbrace\).

Then

\begin{equation}
A_{x,s,y} = A_{x,s,y}^{(1)} + A_{x,s,y}^{(2)}.
\label{eq:supply}\end{equation}

\begin{figure}
\centering
\pandocbounded{\includegraphics[keepaspectratio,alt={\label{fig:procurement}Doses procured by country income level}]{README_files/figure-latex/procurement-1.pdf}}
\caption{\label{fig:procurement}Doses procured by country income level}
\end{figure}

\subsection{Delivery}\label{delivery}

\textbf{These values do not look correct}

Delivery is written as \(h_{s,i,w}^{(j)}\) doses delivered in scenario \(s\), income group \(i\) and week \(w\) for schedule \(j\), which is \(j=1\) for first-dose SSV, \(j=2\) for second-dose SSV, and \(j=2+k\) for \(k=\lbrace 1,...,N^{(boost)}\rbrace\).

The second dose is prioritised over the first, and follows the first by four weeks, so

\[
h_{s,i,w}^{(2)} = h_{s,i,w-4}^{(1)}.
\]

Available doses are \(K_{s,i,w}\), the cumulative doses allocated to income group \(i\) by week \(w\), minus doses given so far. First doses stop being given once \(L_i/2\) is reached.

\[
h_{s,i,w}^{(1)} =
\max\left( 0,
\min\left(  K_{s,i,w} - h_{s,i,w}^{(2)} - \sum_{j=1}^2 \sum_{k=1}^{w-1} h_{s,i,k}^{(j)},
L_i/2 - \sum_{k=1}^{w-1} h_{s,i,k}^{(1)}
\right)
\right)
\]

Booster doses are given for \(j=2+k\) for \(k=\lbrace 1,...,N^{(boost)}\rbrace\):

\[
h_{s,i,w}^{(2+k)} = h_{s,i,w-52k}^{(2)}.
\]

Total doses are therefore

\[
h_{s,i,w} = \sum_{j=1}^{2+N^{(boost)}}h_{s,i,w}^{(j)}.
\]

\begin{figure}
\centering
\pandocbounded{\includegraphics[keepaspectratio,alt={\label{fig:scendelivery}Cumulative vaccine coverage (second SSV dose) by country income level}]{README_files/figure-latex/scendelivery-1.pdf}}
\caption{\label{fig:scendelivery}Cumulative vaccine coverage (second SSV dose) by country income level}
\end{figure}

\section{BPSV delivery}\label{bpsv-delivery}

\subsection{Timing}\label{timing-1}

The duration of the Phase III trial is \(W_3^{(B)} = 18\) weeks. The time to manufacturing transition is \(I_R = 12\) weeks, and the time to manufacturing scale-up \(C_R = 10\) weeks; these are the same as the reserved-capacity times for SSV.

Facility transition occurs in week 1. Thus manufacturing begins in week \(1+I_R = 13\) and dose distribution begins in week \(1+W_3^{(B)} = 19\).

\subsection{Production}\label{production-1}

The number of doses, in billions, that are made in week \(w\) is:

\[
Z_w = \begin{cases}0 & w < I_R \\ 
\frac{1}{52}\frac{w-I_x+1}{C_x}M_{x,s} & w\in[I_R, I_R+C_R) \\ 
\frac{1}{52}M_{x,s}  & w-1\geq I_R+C_xR
\end{cases}
\]

\begin{figure}
\centering
\pandocbounded{\includegraphics[keepaspectratio,alt={\label{fig:bpsvsupply}BPSV doses made available from manufacturing per scenario. Weeks are in reference to the sequencing of the pathogen.}]{README_files/figure-latex/bpsvsupply-1.pdf}}
\caption{\label{fig:bpsvsupply}BPSV doses made available from manufacturing per scenario. Weeks are in reference to the sequencing of the pathogen.}
\end{figure}

\subsection{Allocation}\label{allocation-1}

Doses are all allocated in proportion to the eligible population.

\begin{figure}
\centering
\pandocbounded{\includegraphics[keepaspectratio,alt={\label{fig:bpsvprocurement}BPSV doses procured by country income level}]{README_files/figure-latex/bpsvprocurement-1.pdf}}
\caption{\label{fig:bpsvprocurement}BPSV doses procured by country income level}
\end{figure}

\subsection{Delivery}\label{delivery-1}

\textbf{These values do not look correct}

\begin{figure}
\centering
\pandocbounded{\includegraphics[keepaspectratio,alt={\label{fig:bpsvdeliveryplot}BPSV vaccine coverage by country income level}]{README_files/figure-latex/bpsvdeliveryplot-1.pdf}}
\caption{\label{fig:bpsvdeliveryplot}BPSV vaccine coverage by country income level}
\end{figure}

\section{Parameter samples}\label{parameter-samples}

\pandocbounded{\includegraphics[keepaspectratio]{README_files/figure-latex/parsamples-1.pdf}} \pandocbounded{\includegraphics[keepaspectratio]{README_files/figure-latex/parsamples-2.pdf}} \pandocbounded{\includegraphics[keepaspectratio]{README_files/figure-latex/parsamples-3.pdf}} \pandocbounded{\includegraphics[keepaspectratio]{README_files/figure-latex/parsamples-4.pdf}} \pandocbounded{\includegraphics[keepaspectratio]{README_files/figure-latex/parsamples-5.pdf}} \pandocbounded{\includegraphics[keepaspectratio]{README_files/figure-latex/parsamples-6.pdf}} \pandocbounded{\includegraphics[keepaspectratio]{README_files/figure-latex/parsamples-7.pdf}} \pandocbounded{\includegraphics[keepaspectratio]{README_files/figure-latex/parsamples-8.pdf}} \pandocbounded{\includegraphics[keepaspectratio]{README_files/figure-latex/parsamples-9.pdf}} \pandocbounded{\includegraphics[keepaspectratio]{README_files/figure-latex/parsamples-10.pdf}} \pandocbounded{\includegraphics[keepaspectratio]{README_files/figure-latex/parsamples-11.pdf}} \pandocbounded{\includegraphics[keepaspectratio]{README_files/figure-latex/parsamples-12.pdf}} \pandocbounded{\includegraphics[keepaspectratio]{README_files/figure-latex/parsamples-13.pdf}} \pandocbounded{\includegraphics[keepaspectratio]{README_files/figure-latex/parsamples-14.pdf}} \pandocbounded{\includegraphics[keepaspectratio]{README_files/figure-latex/parsamples-15.pdf}} \pandocbounded{\includegraphics[keepaspectratio]{README_files/figure-latex/parsamples-16.pdf}} \pandocbounded{\includegraphics[keepaspectratio]{README_files/figure-latex/parsamples-17.pdf}} \pandocbounded{\includegraphics[keepaspectratio]{README_files/figure-latex/parsamples-18.pdf}} \pandocbounded{\includegraphics[keepaspectratio]{README_files/figure-latex/parsamples-19.pdf}} \pandocbounded{\includegraphics[keepaspectratio]{README_files/figure-latex/parsamples-20.pdf}} \pandocbounded{\includegraphics[keepaspectratio]{README_files/figure-latex/parsamples-21.pdf}} \pandocbounded{\includegraphics[keepaspectratio]{README_files/figure-latex/parsamples-22.pdf}} \pandocbounded{\includegraphics[keepaspectratio]{README_files/figure-latex/parsamples-23.pdf}} \pandocbounded{\includegraphics[keepaspectratio]{README_files/figure-latex/parsamples-24.pdf}} \pandocbounded{\includegraphics[keepaspectratio]{README_files/figure-latex/parsamples-25.pdf}} \pandocbounded{\includegraphics[keepaspectratio]{README_files/figure-latex/parsamples-26.pdf}}

\section{Contributors}\label{contributors}

Model: Peter Windus, Andy Torkelson

Data: Peter Windus, Andy Torkelson, Damian Walker

Documentation: Peter Windus, Andy Torkelson, Rob Johnson

R code: Rob Johnson

\renewcommand\refname{References}
\bibliography{../../epi.bib}

\end{document}
